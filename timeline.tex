\section{Roadmap and Timeline} \label{sec:timeline}

Table~\ref{tab:ops-timeline} shows the Operations timeline and provides nominal date ranges for the various elements of the Early Science Program.
The date ranges are derived from the Rubin ``Celebratory Milestones'', which are  published monthly on the Rubin Project website\footnote{\url{ls.st/dates}}. 
For completed milestones the delivered date is given. 
Milestone dates are given as min -- max ranges to indicate the associated uncertainty. 
Typically the near date corresponds to the current Project forecast, plus any additional operational uncertainty.
The late date corresponds approximately to the current Project late date plus any additional operational uncertainty.
The shaded region contains roughly 80\% of the probability, while the lefthand edge of the shaded range indicates the earliest date the milestone could be reached. 
The darker regions give a very rough indication of the +/-1 sigma error bars.
Over the course of the commissioning period, we expect these data ranges to shrink as our understanding of the remaining schedule uncertainty improves. 
However, there is still the possibility of the assumptions underlying these distributions being wrong; this is just our best estimate at the current time.

% Auto-generated table - do not edit 
\begin{table}
\centering
\fontsize{6}{10}\selectfont 
\setlength{\tabcolsep}{1.2pt} % Default value: 6pt
{\renewcommand{\arraystretch}{1.2}
\begin{tabular}{|
      p{37mm}|p{25mm}|
%      p{1mm} !{\color{gray}\vrule} p{1mm} !{\color{gray}\vrule} p{1mm} !{\color{gray}\vrule}p{1mm} !{\color{gray}\vrule} p{1mm} !{\color{gray}\vrule}p{1mm} !{\color{gray}\vrule} p{1mm} !{\color{gray}\vrule}p{1mm} !{\color{gray}\vrule} p{1mm} !{\color{gray}\vrule} p{1mm} !{\color{gray}\vrule} p{1mm} !{\color{gray}\vrule} p{1mm}|  % 2024
      p{1mm} !{\color{gray}\vrule} p{1mm} !{\color{gray}\vrule} p{1mm} !{\color{gray}\vrule}p{1mm} !{\color{gray}\vrule} p{1mm} !{\color{gray}\vrule}p{1mm} !{\color{gray}\vrule} p{1mm} !{\color{gray}\vrule}p{1mm} !{\color{gray}\vrule} p{1mm} !{\color{gray}\vrule} p{1mm} !{\color{gray}\vrule} p{1mm} !{\color{gray}\vrule} p{1mm}|  % 2025
            p{1mm} !{\color{gray}\vrule} p{1mm} !{\color{gray}\vrule} p{1mm} !{\color{gray}\vrule}p{1mm} !{\color{gray}\vrule} p{1mm} !{\color{gray}\vrule}p{1mm} !{\color{gray}\vrule} p{1mm} !{\color{gray}\vrule}p{1mm} !{\color{gray}\vrule} p{1mm} !{\color{gray}\vrule} p{1mm} !{\color{gray}\vrule} p{1mm} !{\color{gray}\vrule} p{1mm}|  % 2026
                  p{1mm} !{\color{gray}\vrule} p{1mm} !{\color{gray}\vrule} p{1mm} !{\color{gray}\vrule}p{1mm} !{\color{gray}\vrule} p{1mm} !{\color{gray}\vrule}p{1mm} !{\color{gray}\vrule} p{1mm} !{\color{gray}\vrule}p{1mm} !{\color{gray}\vrule} p{1mm} !{\color{gray}\vrule} p{1mm} !{\color{gray}\vrule} p{1mm} !{\color{gray}\vrule} p{1mm}|  % 2027
                        p{1mm} !{\color{gray}\vrule} p{1mm} !{\color{gray}\vrule} p{1mm} !{\color{gray}\vrule}p{1mm} !{\color{gray}\vrule} p{1mm} !{\color{gray}\vrule}p{1mm} !{\color{gray}\vrule} p{1mm} !{\color{gray}\vrule}p{1mm} !{\color{gray}\vrule} p{1mm} !{\color{gray}\vrule} p{1mm} !{\color{gray}\vrule} p{1mm} !{\color{gray}\vrule} p{1mm}|  % 2028
}
 
    \hline
 \multicolumn{50}{|l|}{\fontsize{9}{12}\selectfont \color{RubinDarkTeal}\textbf{Rubin Operations Survey and Data Release Timeline}  } 
      \\ 
      
 \multicolumn{50}{|l|}{{\fontsize{7}{12}\selectfont \textbf{Nominal LSST Start Date: }October 2025}} 
      \\ \hline
%
 \textbf{Event}  &  \textbf{Date Range}
   %  & \multicolumn{12}{c|}{\textbf{2024}}
     & \multicolumn{12}{c|}{\textbf{2025}}
     & \multicolumn{12}{c|}{\textbf{2026}}
     & \multicolumn{12}{c|}{\textbf{2027}}
     & \multicolumn{12}{c|}{\textbf{2028}}
           \\ \hline
%%     
%\tiny  Data Preview 0.1  &  \tiny  Delivered Jun 2021   
%  %   &&&&&&&&&&&&   
%     &&&&&&&&&&&&  
%     &&&&&&&&&&&& 
%     &&&&&&&&&&&&       
%     &&&&&&&&&&&&       
%\\  \arrayrulecolor{lightgray}\hline
%
%\tiny  Data Preview 0.2  &  \tiny  Delivered  Jun 2022   
%  %   &&&&&&&&&&&&   
%     &&&&&&&&&&&&  
%     &&&&&&&&&&&& 
%     &&&&&&&&&&&&       
%     &&&&&&&&&&&&       
%\\\hline
%
%\tiny  Data Preview 0.3  &  \tiny  Delivered  Jun 2023
% %    &&&&&&&&&&&&   
%     &&&&&&&&&&&&  
%     &&&&&&&&&&&& 
%     &&&&&&&&&&&&       
%     &&&&&&&&&&&&     
%\\\hline
\tiny  Data Preview 0.1/2/3  (DP0)&  \tiny  Delivered  Jun 2023
 %    &&&&&&&&&&&&   
     &&&&&&&&&&&&  
     &&&&&&&&&&&& 
     &&&&&&&&&&&&       
     &&&&&&&&&&&&     
\\\hline

\tiny Data Preview 1 (DP1)   &    \tiny Jun 2025 -- Jul 2025      
 %  & & & & &  & &  &&& &  &  % 2024
     &  &  &  &    &  &\cellcolor{TLOrange2}  &  \cellcolor{TLOrange2} &&&&&   % 2025
       & & && &&&&&&&&  % 2026
     & & &&&&&&&&&&   % 2027
     & & &&&&&&&&&&   % 2028
\\\hline

\tiny Rubin First Light  (RFL)    &    \tiny  Jul 2025      
%     & & &&&&&&&&&&   % 2024
   &   &  & &    & & &  \cellcolor{TLOrange2}   & & &&&   % 2025
       & & &&&&&&&&&&   % 2026
     & & &&&&&&&&&&   % 2027
     & & &&&&&&&&&&   % 2028
\\\hline

\tiny First Look LSSTCam Data (FLL)    &    \tiny  Jul 2025      
%     & & &&&&&&&&&&   % 2024
   &   &  & &    & & &  \cellcolor{TLOrange2}   & \cellcolor{TLOrange2}   & &&&   % 2025
       & & &&&&&&&&&&   % 2026
     & & &&&&&&&&&&   % 2027
     & & &&&&&&&&&&   % 2028
\\\hline
%
\tiny Rubin First Alerts  (RFA)   &    \tiny  Jul 2025 -- Sep 2025     
%     & & &&&&&&&&&&   % 2024
   &   &  & &    & & &  \cellcolor{TLOrange3}   & \cellcolor{TLOrange2}   & \cellcolor{TLOrange3} &&&   % 2025
       & & &&&&&&&&&&   % 2026
     & & &&&&&&&&&&   % 2027
     & & &&&&&&&&&&   % 2028
\\\hline

\tiny Start of Operations (OPS)    &    \tiny  Sep 2025 -- Oct 2025      
   %  &&&&&&&&&&&&   % 2024
     &   &&&& &  &  &   & \cellcolor{TLOrange3} &  \cellcolor{TLOrange2}  &&   % 2025 
     & & &&&&&&&&&&   % 2026
     & & &&&&&&&&&&   % 2027
     & & &&&&&&&&&&   % 2028
\\\hline

\tiny Start of LSST    (SVY)     &    \tiny Sep 2025 -- Nov 2025      
 %   &&&&&&&&&&&&   % 2024
     &   &&&& &  &  &   & \cellcolor{TLOrange3} &  \cellcolor{TLOrange2}  & \cellcolor{TLOrange3}  &   % 2025 
     & & &&&&&&&&&&   % 2026
     & & &&&&&&&&&&   % 2027
     & & &&&&&&&&&&   % 2028
\\\hline
% 
\tiny Start Regular Alert Production (RAP)  &    \tiny  Sep 2025 -- Nov 2025      
 %   &&&&&&&&&&&&   % 2024
     &   &&&& &  &  &   & \cellcolor{TLOrange3} &  \cellcolor{TLOrange2}  & \cellcolor{TLOrange3}  &   % 2025 
     & & &&&&&&&&&&   % 2026
     & & &&&&&&&&&&   % 2027
     & & &&&&&&&&&&   % 2028
\\\hline

\tiny Data Preview 2  (DP2)  &    \tiny  Mar 2026 -- May 2026      
 %   & & &&&&&&&&&&   % 2024
    & & &&&&&&&&&& % 2025
    & &&\cellcolor{TLOrange3} &\cellcolor{TLOrange2} &\cellcolor{TLOrange3}  &&&&&&&   % 2026
     & & &&&&&&&&&&   % 2027
     & & &&&&&&&&&&   % 2028
\\\hline

\tiny Data Release  1 (DR1)  &    \tiny Sep 2026 -- Jan 2027      
%   & & &&&&&&&&&&   % 2024
   & & &&&&&&&&&&   % 2025
   & & &&&  &  && & \cellcolor{TLOrange3} & \cellcolor{TLOrange3} & \cellcolor{TLOrange2} & \cellcolor{TLOrange2}    % 2026
   & \cellcolor{TLOrange3}  & &&&&&&&&&&   % 2027
   & & &&&&&&&&&&   % 2028
\\\hline

\tiny Data Release  2 (DR2)  &    \tiny Sep 2027 -- Jan 2028      
 %  & & &&&&&&&&&&   % 2024
   & & &&&&&&&&&&   % 2025
   &  & &&&&&&&&&&   % 2026
   & & &&& & && & \cellcolor{TLOrange3} & \cellcolor{TLOrange3} & \cellcolor{TLOrange2} & \cellcolor{TLOrange2}    % 2027
   &  \cellcolor{TLOrange3} & &&&&&&&&&&   % 2028
\\\hline
\tiny Data Release  3 (DR3)  &    \tiny Sep 2028 -- Nov 2028      
%   & & &&&&&&&&&&   % 2024
   & & &&&&&&&&&&   % 2025
   & & &&&&&&&&&&   % 2026
   & & &&&&&&&&&&   % 2027
   & & &&& & &  & & \cellcolor{TLOrange3}  & \cellcolor{TLOrange2}  &  \cellcolor{TLOrange3} &   % 2028
 \\ \arrayrulecolor{black}\hline\hline

% Months annotation
 \multicolumn{2}{|l|}{}  
%    &  \scalebox{.7}  J  & \scalebox{.7} F  & \scalebox{.7} M &  \scalebox{.7}  A & \scalebox{.7}  M & \scalebox{.7}  J & \scalebox{.7}  J & \scalebox{.7}  A & \scalebox{.7}  S & \scalebox{.7}  O& \scalebox{.7}  N & \scalebox{.7}  D    % 2024
    &  \scalebox{.7}  J & \scalebox{.7}  F  & \scalebox{.7}  M &  \scalebox{.7}  A & \scalebox{.7}  M & \scalebox{.7}  J & \scalebox{.7}  J & \scalebox{.7}  A & \scalebox{.7}  S & \scalebox{.7}  O& \scalebox{.7}  N & \scalebox{.7}  D    % 20245 
        &  \scalebox{.7}  J & \scalebox{.7}  F  & \scalebox{.7}  M &  \scalebox{.7}  A & \scalebox{.7}  M & \scalebox{.7}  J & \scalebox{.7}  J & \scalebox{.7}  A & \scalebox{.7}  S & \scalebox{.7}  O& \scalebox{.7}  N & \scalebox{.7}  D    % 2026
       &  \scalebox{.7}  J & \scalebox{.7}  F  & \scalebox{.7}  M &  \scalebox{.7}  A & \scalebox{.7}  M & \scalebox{.7}  J & \scalebox{.7}  J & \scalebox{.7}  A & \scalebox{.7}  S & \scalebox{.7}  O& \scalebox{.7}  N & \scalebox{.7}  D    % 2027
       &  \scalebox{.7}  J & \scalebox{.7}  F  & \scalebox{.7}  M &  \scalebox{.7}  A & \scalebox{.7}  M & \scalebox{.7}  J & \scalebox{.7}  J & \scalebox{.7}  A & \scalebox{.7}  S & \scalebox{.7}  O& \scalebox{.7}  N & \scalebox{.7}  D     % 2028
 \\ \arrayrulecolor{black}\hline\hline
     
\end{tabular}}
\caption{Rubin Operations Key Milestones for Early Science}
\label{tab:ops-timeline}
\end{table}



The next key milestone in the Early Science Program is the release of DP1, expected sometime between June and July 2025.
The  start of Rubin Operations is currently expected to be sometime between September and October 2025.
The timing of the Commissioning observations and their release to the community can only be projected to within a few months at the time of writing.

Table ~\ref{tab:ops-timeline} will continue to be refined and updated in future version of this document as the Early Science Program progresses.