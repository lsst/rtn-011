% Merge summary and introduction 
\section{Introduction}

Community expectations for early science with Rubin are high due to the transformative nature of the LSST data and the densely-sampled observations planned during the commissioning period.
Rubin Observatory's \emph{Early Science Program} is designed to provide Rubin data rights holders with access to the data products and services necessary to produce high-impact early science during time between commissioning through, and including, the first data release, Data Release 1 (DR1). 

\subsection{Definition of Early Science}  \label{ssec:defn}
Early Science is defined as any science enabled by Rubin for its community through and including the first LSST Data Release, DR1.

\subsection{Early Science scenarios } \label{ssec:scenarios}

Recent planning on the construction project has led to a reduced amount of on-sky time in commissioning, including a reduction in the time dedicated to final science validation of the as-built system compared to earlier draft plans.
The total amount science validation time currently planned during commissioning is 8 weeks.
As Rubin construction moves through the challenging phase of System Integration, Test and Commissioning (SIT-Com), on--sky time could be further reduced.

The Operations team is tracking the progress of the commissioning activities as they relate to Early Science opportunities to ensure that the community has timely access to science-ready data products while the survey begins its relentless coverage of the sky leading to DR1.
We broadly envisage two possible scenarios emerging from the commissioning phase of the construction project: 

\begin{itemize}
\item \textbf{Scenario A}:
The full commissioning plan comprising system optimization and science validation is successfully executed as planned. 
Rubin Operations then carries out an Operations Rehearsal and Operations Readiness Review (ORR) to effectively conduct a ``full dress rehearsal'' of science operations and demonstrate the readiness of the Operations team to execute the 10-year survey. 
Data collected during commissioning and the SV Surveys is reprocessed to produce DP2, which will be released 6 months following the completion of the SV Surveys, see \S~\ref{tab:dr-one-products}.

\item \textbf{Scenario B}:
On-sky time in commissioning is reduced as the construction work  draws to an end, resulting in the SV surveys not being completed prior to the end of the construction phase.
The Operations team would spend up to 3 months prior to commencing the 10-year LSST survey carrying out any remaining SV Survey observations and releasing DP2 with the same data products as planned in Scenario A.

As per Scenario A, data collected during commissioning and the SV Surveys is reprocessed to produce DP2 and an Operations Readiness Review carried out to demonstrate readiness to execute the 10-year survey. 

\end{itemize}

A key point to note is that the contents of DP2 will be the same irrespective of which scenario materializes.
Only the timing of the release of DP2 and the start of the 10-year survey are different between the two scenarios.
In both scenarios it is assumed that the Rubin Construction project delivers an integrated system that can capture, transfer and process science-grade data at the time Operations begins.
(Note that the First Light observations needed for DP1 must be taken by prior to project completion, such that DP1 is the same in each scenario.)
Both scenarios will include alert generation of some type, with the major distinction being the relative availability of templates in time, sky position, and filter.

These two scenarios presented are current as of December 2022, however are subject to change commissioning program emerges and is executed.
At some future point, a single option will be adopted and executed, and at that time, the details will be more fully specified.

