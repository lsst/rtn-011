% Merge summary and introduction 
\section{Rubin Early Science Program}

Community expectations for early science with Rubin are high due to the transformative nature of the LSST data and the densely-sampled observations planned during the commissioning period.
Rubin Observatory's \emph{Early Science Program} is designed to provide Rubin data rights holders with access to the data products and services necessary to produce high-impact early science during time between commissioning through, and including, the first data release, Data Release 1 (DR1). 

\subsection{Definition of Early Science}  \label{ssec:defn}
Early Science is defined as any science enabled by Rubin for its community through and including the first LSST Data Release, DR1.
 This includes the commissioning period and the first year of survey operations.

\subsection{Motivation for an Early Science Program}  \label{ssec:motivation}

The Early Science program is motivated by the desire to:
\begin{itemize}
\item enable high-impact science as early as possible;
\item provide early access to both static-sky and time-domain science-ready data products to support the community to prepare in advance of the first survey data release;
\item enable early time-domain astronomy via Alert Production; and 
\item help drive development of Rubin operations capabilities prior to survey start and prepare the team to be operations-ready.
\end{itemize}

\subsection{Elements of the Early Science Program}

The Early Science Program consists of the following elements:
\begin{itemize}
	\item A series of three \textbf{Data Previews (DP)}, DP0, DP1 and DP2,  based on either simulated LSST-like data or data taken during the Rubin Observatory commissioning period with the LSST Science Camera (LSSTCam). 
	\item A world-public \textbf{stream of Alerts} from transient, variable, and moving sources that will be scaled up continuously during commissioning and the first year of the survey. 
	\item  \textbf{Template generation}, both prior to the start of regular survey operations based on data collected during the commissioning period with LSSTCam, and incrementally during the first year of regular survey operations  to maximize the number of templates available for Alert Production in year 1. 
	\item \textbf{LSST Data Release 1 (DR1)}, which will be based on the Data Release Processing (DRP) of the first six months of LSST data.
\end{itemize}


\subsection{Early Science scenarios } \label{ssec:scenarios}

The Operations team is tracking the progress of the commissioning activities (\S~\ref{sec:commissioning}) as they relate to Early Science opportunities to ensure that the community has timely access to science-ready data products while the survey begins its relentless coverage of the sky leading to DR1.
We broadly envisage two possible scenarios emerging from the commissioning phase of the construction project: 

\begin{itemize}
\item \textbf{Scenario A}:
Rubin Observatory is ready to execute the 10-year LSST at the completion of the construction project. 
The operations team may decide to first conduct a \textit{full dress rehearsal} of science operations to demonstrate team readiness prior to commencing execution of the LSST.
In this scenario, we expect this rehearsal to take a few days to no longer than two weeks. 

\item \textbf{Scenario B}:
Prior to commencing the 10-year LSST, the operations team decides to spend up to a maximum of 2 months collecting more on-sky data to complement and extend the datasets collected during commissioning. 
This additional data would serve the dual goals of 1) ensuring that we are fully prepared to start the 10-year survey and 2) providing the community with an exquisite early dataset to work with before DR1. 
As per Scenario A, the operations team may decide to first conduct a \textit{full dress rehearsal} of science operations to demonstrate team readiness prior to commencing execution of the LSST. 
\end{itemize}
As the survey begins, all science-grade data collected during the commissioning System Optimization period and subsequent Science Validation Surveys, \S~\ref{sec:commissioning}, is reprocessed to produce the final Data Preview, DP2, which will be released 6 months following the completion of the Science Validation Surveys.

In both scenarios it is assumed that the Rubin Construction project delivers an integrated system that can capture, transfer and process science-grade data at the time operations begins.
Both scenarios will include alert generation of some type, with the major distinction being the relative availability of templates in time, sky position, and filter. 
The First Light observations that form the basis of DP1 must be taken and analysed in order to support construction completeness, meaning that DP1 is identical in each scenario.
The DP2 data products will be the same irrespective of which scenario materializes; only the timing of the release of DP2 and the start of the 10-year survey are different between the two scenarios.
These two scenarios presented are current as of December 2022, but are subject to change as the commissioning program is executed.
At some future point, a single option will be adopted and executed, and at that time, the details will be more fully specified.

