\section{Introduction}

This note describes the plan for ensuring the Rubin community will have the data products and services necessary to produce high-impact early science during the time between commissioning and the first LSST Data Release, DR1.  
Community expectations for early science are high due to the transformative nature of the Rubin data and the extensive amount of on-sky time planned in commissioning and science validation.

\subsection{Definition of Early Science}  \label{ssec:defn}

Early Science (ES) is defined as any science enabled by Rubin for its community through and including the first data release, \drone.

DR1 is the first planned release of static-sky data products. 
It will be based on the first 6 months of data and released 12 months after the start of full survey operations.  
Alerts to transient, variable and moving objects are produced by the LSST Prompt Processing pipelines via Difference Image Analysis (DIA), which creates difference images by subtracting a template image \footnote{Templates are transient-free coadds produced by stacking previous images such as to remove the flux of any transient or moving object.} from each new image acquired, and then identifying sources in the difference image. 
Without templates for a given pointing and filter, there can be no Alerts.
Templates are produced during Data Release Processing (DRP) and initially released as part of DR1, so in year 1 there will be no templates from which to generate alerts, and hence no Alerts.
The Early Science Program has been conceived with the goal of providing science-ready data products to the Rubin community during the time between commissioning and the first LSST Data Release, DR1.

\subsection{Elements of the Early Science Program}

The \esp consists of the following elements:
\begin{itemize}
	\item A series of ``Data Previews'' based on reprocessed data taken during the Rubin Observatory commissioning period. The first, Data Preview 1 (DP1) will be based on ComCam data and the second, Data Preview (DP2) based on LSSTCam data. 
	\item Template generation prior to the start of regular survey operations based on data collected during the commissioning period with the LSST Camera (LSSTCam) to maximize the number of templates available for Alert Production at the start of the 10-year survey.
	\item Incremental generation of templates during the first year of survey operations, as and when sufficient images passing quality cuts have been acquired. These templates will then be used immediately during year 1 to generate alerts, thus steadily increasing the number of Alerts generated over the course of year 1. 
\end{itemize}

Rubin pre-operations is working with the construction project to provide early access to static-sky data products via Data Previews. 
The first of these, Data Preview 0 (DP0), was released to a group of early adopters from the community in June 2022 and is based on the DESC DC2 simulated dataset, \citep{2021ApJS..253...31L}. 
Subsequent Data Previews, DP1 and DP2 will be based on the data acquired during  commissioning from ComCam and LSSTCam Science Validation Surveys (SV Surveys) respectively, together with any additional science-quality data taken throughout the full commissioning period.

A key component of the \esp is the generation of templates from both the commissioning data and data collected during the first year of the survey.
In full survey operations, template images for difference image analysis and alert generation are constructed as part of the annual DRP.
In order to support alert generation in year 1, Rubin will generate templates from all science-grade data taken during commissioning to provide an initial template library at the start of the 10-year survey, and the incrementally generate templates during year 1 using the best images available and covering as much sky in as many filters as possible.
Details of the current strategy for alert generation  with incremental templates are given in  \S~\ref{sec:pp}.


\subsection{Early Science scenarios } \label{ssec:scenarios}

Recent planning on the construction project has led to a reduced amount of on-sky time in commissioning, including a reduction in the time dedicated to final science validation of the as-built system compared to earlier draft plans.
The total amount science validation time currently planned during commissioning is 8 weeks. 
As Rubin construction moves through the challenging phase of System Integration, Test and Commissioning (SIT-COM), on--sky time could be further reduced.

The Operations team is tracking the progress of the commissioning activities as they relate to \es opportunities to ensure that the community has prompt access to science-ready data products while the survey begins its relentless coverage of the sky leading to DR1.

We broadly envisage two possible scenarios emerging from the commissioning phase of the construction project: 

\begin{itemize}
\item \textbf{Scenario A}:
The full commissioning plan comprising system optimization and science validation is successfully executed as planned. 
The Operations team will then carry out an Operations Readiness Review (ORR) to effectively conduct a full dress rehearsal of science operations and demonstrate the readiness of the Operations team to execute the 10-year survey. 
Data collected during commissioning and the SV Surveys is reprocessed to produce DP2, which will be released 6 months following the completion of the SV Surveys, see \S~\ref{tab:dr-one-products}.

\item \textbf{Scenario B}:
On-sky time in commissioning is reduced further resulting in the SV Surveys not being completed prior to the ORR. 
The Operations team will spend up to 3 months prior to commencing the 10-year survey to complete the SV Surveys as planned. 
As per Scenario A, data collected during commissioning and the SV Surveys is reprocessed to produce DP2 and an Operations Readiness Review carried out to demonstrate readiness to execute the 10-year survey. 

\end{itemize}

A key point to note is that the contents of DP2 will be unchanged irrespective of which scenario materializes.  
Only the timing of the release of DP2 and the start of the 10-year survey are different between the two scenarios. 
In both scenarios it is assumed that the Rubin Construction project delivers an integrated system that can capture, transfer and process science-grade data at the time Operations begins.
Both scenarios will include alert generation of some type, with the major distinction being the relative availability of templates in time, sky position, and filter.


These two scenarios presented are current as of October 2022, but since this document is ``living,'' we expect the plans to mature as we approach full survey operations and the commissioning program emerges and is executed.
At some future point, a single option will be adopted and executed, and at that time, the details will be more fully specified.
