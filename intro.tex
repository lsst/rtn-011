% Merge summary and introduction
\section{Rubin Early Science Program}

Community expectations for early science with Rubin are high due to the transformative nature of the LSST data and the densely-sampled observations planned during the commissioning period.
Rubin Observatory's \emph{Early Science Program} was conceived to provide Rubin data rights holders with early access to the data products and services necessary to produce high-impact science during the period from the beginning of commissioning through the conclusion of the first year of the LSST.

\subsection{Definition of Early Science}
\label{ssec:defn}
Early Science is defined as any science enabled by Rubin for its community during the period from the beginning of commissioning to the conclusion of the first year of the LSST survey.
This includes the commissioning period and the first year of survey operations.

\subsection{Motivation for an Early Science Program}  \label{ssec:motivation}

The Early Science program is motivated by the desire to:
\begin{itemize}
\item enable high-impact science with Rubin Observatory data as early as possible;
\item provide early access to both static-sky and time-domain science-ready data products to support the community to prepare for science with the LSST;
\item enable early time-domain astronomy via early Alert Production; and
\item help drive the development of Rubin operations capabilities prior to survey start and prepare the team to be operations-ready.
\end{itemize}

\subsection{Elements of the Early Science Program}

The Early Science Program consists of the following elements:
\begin{itemize}
	\item A series of three \textbf{Data Previews (DP)}, DP0, DP1 and DP2,  based on either simulated LSST-like data or data taken during the Rubin Observatory commissioning period.
	\item A world-public \textbf{stream of Alerts} from transient, variable, and moving sources that will be scaled up continuously during commissioning and the first year of the survey.
	\item  \textbf{Template generation}, both prior to the start of regular survey operations based on data collected during the commissioning period with LSSTCam, and incrementally during the first year of regular survey operations  to maximize the number of templates available for Alert Production in year 1.
	\item \textbf{LSST Data Release 1 (DR1)}, which will be based on the Data Release Processing (DRP) of the first six months of LSST data following the baseline survey strategy.
\end{itemize}

%
\subsection{Transition to Operations and Early Science}
\label{ssec:transition}

The Rubin Construction project will deliver an integrated system that can capture, transfer and process science-grade data, following which, the Construction project will be declared complete and Operations will begin.
The Operations team is tracking the progress of the commissioning activities (\S~\ref{sec:commissioning}) to identify opportunities for Early Science and address the goals described in \S~\ref{ssec:motivation}.
The data collected as part of the SV surveys in Construction will serve the dual goals of 1) ensuring that Operations is fully prepared to start the 10-year survey and 2) providing the community with an exquisite early dataset to work with while the survey begins its relentless coverage of the sky leading to DR1.
All Early Science data products are opportunistic on the commissioning activities, meaning that a detailed description will only be possible once the commissioning data have been acquired and analyzed.
Consequently, while the Operations team will do its best to deliver the maximum of early data, data product types and services, any statement on the contents of the Data Previews, early Alert Stream and supporting services is subject to change up until the release date.


\subsection{Factors Impacting the Early Science Program}
\label{ssec:impact}

Factors affecting the schedule and contents of the Early Science program can be broadly grouped into technical considerations and policy considerations.
Technical considerations include:
\begin{itemize}
\item The operational status of the observatory and progress of system integration and test activities in commissioning;
\item The nature and quality of the data collected during commissioning;
\item The readiness of the data processing pipelines, and data distribution and access services.
\end{itemize}
Policy factors include:
\begin{itemize}
\item The 30-day embargo on all pixel data during commissioning;
\item The 80-hour embargo on all pixel data throughout the full duration of the LSST;
\item The  construction security review, which must be successfully completed prior to the release of any Prompt data products;
\item The Rubin First Look (RFL) media event, currently expected in June 2025 and before which  no Rubin image data may be released. See~\citeds{rtn-083} for details.
\end{itemize}
In this document, the term ``stretch goal'' will be used to describe cases where any uncertainty is due to a technical or scientific consideration and ``TBD'' (To Be Decided) will be used when the influencing factor is of a policy nature.
