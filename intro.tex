\section{Introduction}

This  note describes the plan for ensuring the Rubin community will have the data products and services necessary to produce early science during the time between commissioning and the release of the year 1 data.

\subsection{Definition of Early Science}  \label{ssec:defn}

Early Science (ES) is defined as any science enabled by Rubin for its community through and including the first data release, \drone.
DR1 will be based on the first 6 months of data and is scheduled for release 12 months  after the start of full survey operations, \citep{RDO-011}.
Community expectations for early science are high due to the transformative nature of the Rubin data and the extensive amount of on-sky time planned in commissioning and science validation.

\subsection{Motivations for an Early Science Program}

The \esp is motivated by the fact that in the current baseline there will be no science-ready data products released to the community before \drone.
This is due to the fact that the templates are needed for Difference Imaging are produced in Data Release Production (DRP) and initially released as part of \drone.
Alerts cannot be issued until a template image for a given field and filter is available, hence  in the current baseline, no alerts can be issued until after \drone.

The success of \es then depends on various scenarios coming out of commissioning as we transition into operations, as described in \S~\ref{ssec:scenarios}.

\subsection{Early Science scenarios } \label{ssec:scenarios}

% \TODO{Leanne}{Check and revise the following para.}

Recent planning on the construction project has led to a reduced amount of on-sky time in commissioning, including a reduction in the time dedicated to final science validation of the as-built system compared to earlier draft plans.
The total amount science validation time is currently planned for 8 weeks.
As Rubin construction moves through the challenging phase of System Integration, Test and Commissioning (SIT-COM), on--sky time could be further reduced.
The Operations team is thus planning for various outcomes that might require special attention to producing \es opportunities in the first part of regular operations to ensure the community has prompt access to science-ready data products while the survey begins its relentless coverage of the sky leading to DR1.

\TODO{Leanne}{We can now simplify the following scenarios greatly. Combine B and C into a single Plan B. Then spell out that the content of the DPs and DR1 are unchanged between Plan A and B - the only difference is in the timing of the data releases. Consider dropping the Plan A / B terminology altogether, and just discuss impacts on timing.}

In all cases, it is assumed that the Rubin Construction project delivers an integrated system that can capture, transfer and process science-quality data at the time Operations begins.
Planning will then consider the following three high-level options to ensure \es:
\begin{itemize}
\item \textbf{Option A}:
Science validation is completely successful.
Begin the LSST working towards DR1 while providing \es data products.
\item \textbf{Option B:}
On-sky time in commissioning was reduced such that fewer science-ready data products were produced before the LSST begins but the Rubin system can capture and produce science-quality data.
Instigate a dedicated Early Science observing campaign for a limited period (3-6 months) at the start of full operations that is different to the regular survey cadence.
\item \textbf{Option C:}
The Rubin System  passes the construction completion requirements and can capture and produce science-quality data but the operations team is not yet ready to begin the LSST.
Further shakedown of operations procedures and data taking is required, which will result in a delay to the start of full survey operations.
This option will encompass elements of options B and C as appropriate.
\end{itemize}

All options will include alert generation of some type, with the major distinction being the relative availability of templates in time, sky position, and filter.
The principle aspect of this strategy both  in SIT-COM and year 1 operations is incremental template generation.
In full survey operations, template images for difference image analysis and alert generation are constructed as part of the annual DRP.
In order to support alert generation in year 1, Rubin will incrementally generate templates during SIT-COM and year 1 using the best images available and covering as much sky  in as many filters as possible.
Details of the current strategy for alert generation  with incremental templates are given in  \S~\ref{sec:pp}.

Rubin pre-operations is working with the construction project to provide a series of Data Previews based on the data acquired during the SITCOM \svs, as well as any additional science-quality data taken throughout the full commissioning period \citep{RDO-011}.

These options are current at the time of writing, but as this document is "living",  we expect the plans to mature as we approach full survey operations and the extant SIT-COM program emerges and is executed.
At some future point, a single option will be adopted and executed, and at that time, the details will be more fully specified.
