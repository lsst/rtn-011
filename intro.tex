% Merge summary and introduction 
\section{Rubin Early Science Program}

Community expectations for early science with Rubin are high due to the transformative nature of the LSST data and the densely-sampled observations obtained during the commissioning period.
Rubin Observatory's \emph{Early Science Program} was established to provide Rubin data rights holders with early access to the data products and services needed to produce high-impact science during the period during the period spanning commissioning through Data Release 1 (DR1).

\subsection{Early Science Definitions}
\label{ssec:defn}

The following terms and their definitions are used in this document:
\begin{itemize}

\item\textbf{Commissioning:} The Rubin Construction Commissioning period, from LSSTComCam First Photon on 2024-10-24 through the end of Construction, \cstrcompdate.
\item\textbf{Early Operations:} The period from the start of Rubin Operations, \opsstartdate,  through the first Data Release, DR1.

\item\textbf{Rubin Operations Early Optimization Phase:} The period from the start of Rubin Operations, \opsstartdate, through the start of the LSST.

\item\textbf{Early Science:} Any science enabled by Rubin for its community during the period from the beginning of Commissioning through the first Data Release, DR1.
\end{itemize}

\subsection{Motivation for an Early Science Program} 
 \label{ssec:motivation}
The Early Science program is motivated by the desire to:
\begin{itemize}
\item enable high-impact science with Rubin Observatory data as early as possible;
\item provide early access to both static-sky and time-domain science-ready data products to support the community to prepare for science with the LSST;
\item enable early time-domain astronomy via early Alert Production; and 
\item help drive the development of Rubin operations capabilities prior to survey start and prepare the team to be operations-ready.
\end{itemize}

\subsection{Elements of the Early Science Program}
 \label{ssec:elements}
The Early Science Program consists of the following elements:
\begin{itemize}
	\item A series of three \textbf{Data Previews (DP)}, DP0, DP1 and DP2,  based on either simulated LSST-like data or data taken during the Rubin Observatory commissioning period. 
	\item A world-public \textbf{Stream of Alerts} from transient, variable, and moving sources that will be scaled up continuously during commissioning and the first year of the survey.
	\item  \textbf{Template generation}, both prior to the start of regular survey operations based on data collected during the commissioning period with LSSTCam, and incrementally during the first year of regular survey operations  to maximize the number of templates available for Alert Production in year 1.
	\item \textbf{Nightly Processed Visit Images (PVIs)} and associated Source catalogs of detections up until  Data Release 1. 
	\item \textbf{LSST Data Release 1 (DR1)}, which will be based on the Data Release Processing (DRP) of the first year of LSST data following the baseline survey strategy.
\end{itemize}

%
\subsection{Transition to Operations and Early Science}
\label{ssec:transition}
The Rubin Construction Project was declared substantially complete on \cstrcompdate, marking the delivery of an integrated system capable of capturing, transferring, and processing science-grade images consistent with the Rubin/LSST Science Requirements Document (SRD; \citealt{LPM-17}). 
The following night, the observatory began on-sky operations and entered the Rubin Operations Early Optimization Phase. 
Commissioning demonstrated that the as-built system meets the LSST science requirements, and the Operations team is now focused on achieving stable, repeatable nightly performance while coordinating remaining activities with the Construction team in preparation for full survey operations. 

Data collected during Science Validation surveys provide the foundation for many Early Science data products, offering the community an exceptional early dataset as the survey begins systematic sky coverage leading up to Data Release 1 (DR1). 
These early data products are derived from data obtained during Commissioning (\S~\ref{sec:commissioning}) and Early Operations.
While the Operations team aims to deliver the broadest feasible range of early data products and services, the contents and timelines for Data Previews, the early Alert Stream, and supporting services remain subject to change and reflect the best available estimates at the time of publication.

\subsection{Factors Impacting the Early Science Program}
\label{ssec:impact}
Factors affecting the schedule and contents of the Early Science program can be broadly grouped into technical considerations and policy considerations. 
Technical considerations include:
\begin{itemize}
\item The operational status of the observatory and progress of system integration and test activities in commissioning; 
\item The nature and quality of the data collected during commissioning; 
\item The readiness of the data processing pipelines, and data distribution and access services. 
\end{itemize}
Policy factors include:
\begin{itemize}
\item The 30-day embargo on all pixel data during commissioning (concluded);
\item The 80-hour embargo on all pixel data throughout the full duration of the LSST (current); 
\item The  construction security review, which must be successfully completed prior to the release of any Prompt data products (successfully concluded);
\item The Rubin First Look (RFL) media event \citep{RTN-083},  before which no Rubin image data could be released (concluded). 
\end{itemize}
Of the above, only the 80-hour embargo on all pixel data remains active, and will continue to be, for the duration of Rubin Operations. 

In this document, the term ``stretch goal'' will be used to describe cases where any uncertainty is due to a technical or scientific consideration and ``TBD'' (To Be Decided) will be used when the influencing factor is of a policy nature. 