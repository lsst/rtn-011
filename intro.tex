\section{Introduction}
This technical note describes the initial darft plan for ensuring the Rubin Observatory community will have sufficient data access, data prodcuts, and data analysis tools to produce early science when the full Legacy Survey of Space and Time (LSST)  begins. The start date for full survey operations is uncertain at this time, but we assume for purposes of planning that it will be no earlier than October 1, 2023. This is not an offical project baseline date.

\subsection{What do we mean by Early Science}

Early Science (ES) is any science enabled by Rubin by its community that happens before Data Release 1 (DR1) at the end of the first year of full survey operations. Expectations for early science have largely been high due to the extensive amount of on sky time planned in commissioning and science verification (SV).  

Recent planning on the construction project has led to a reduced amount of on--sky time and SV time.
As Rubin construction moves through the challenging phase of System Integration, Test and Commissioning (SIT-COM), on--sky time could get squeezed more. The Operations team is thus planning for various outcomes that might require special attention to producing ES opportunities in the first part of regular operations to ensure the community has access to exciting data and data sets while the survey begins its relentless coverage of the sky before DR1.

In all cases, it is assumed Rubin Construction hands Rubin Operations a system that can capture, move, and process science quality datai at the time Operations beigns. Planning will then consider three high level options to ensure ES. These are initially described in this document, but the document is "living" and we expect the plans to mature in detail over time as we approach full survey operations and the extant SIT-COM program emerges and is executed. At some point, a single option will be adopted and executed:

\begin{itemize}

\item Plan “A”, SV is completely successful, move quickly to the LSST and DR1,

\item Plan “B”, use Early Science Period (3-6 months) that is different than regular survey operations because on--sky time in SIT-COM is reduced, leading to few science ready data before the LSST begins.

\item Plan “C”, further shakedown of operations procedures and data taking is required even though the initial condition above is satisfied and the Rubin System can capture and produce science quality data.

\end{itemize}


