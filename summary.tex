\section{Summary}

Rubin Observatory is putting in place a dedicated {\it Early Science Program} to ensure high-impact science prior to the release of the first year of \lsst data.
The Rubin Operations team is carrying out a series of ``Data Previews,'' in which LSST pre-cursor data products are prepared, released to the LSST data rights community, and supported.
The goals of the Data Previews are to 1) enable the Operations teams to develop operational capability prior to the start of the LSST, and 2) support the members of the LSST science community as they develop their LSST analyses.
The first Data Preview era, DP0, is already underway: DP0.1 and DP0.2 contain Rubin-processed image and catalog data products derived from simulated LSST images that were generated by the LSST Dark Energy Science Collaboration as the DC2 Virtual Sky Survey \citep{2021ApJS..253...31L}.
A third phase, DP0.3, will focus on catalog-level simulated Solar System science products and is scheduled for release in 2023.
Subsequent Data Previews DP1 and DP2 will contain LSST-like data products generated from the commissioning data taken with LSSTCam, respectively.
The data taken during the first 6 months of the LSST survey will be processed and released as Data Release 1.
DP1, DP2, and DR1 will all include data products for both static-sky science and time-domain science.

Time-domain astronomy is a key component of LSST's four science pillars and is enabled by alerts on LSST detections of transient, variable, and/or moving objects.
Alerts are the only data product that will be immediately available (within 60 seconds of image readout) and publicly shareable, i.e. not subject to a proprietary period \citep{LSE-163},  \citep{RDO-013}.
The worldwide community is actively preparing to process the LSST alert stream and use it to generate groundbreaking scientific results. Additionally, for many science goals, time-sensitive follow-up observations after discovery are crucial to take full advantage of the Rubin data.

A key component of the \esp is the capability to build {\it incremental} templates from on-sky imaging as it becomes available during commissioning and the early phases of the survey.
Such templates will be built periodically as images accumulate to allow for partial alert generation over an incomplete survey footprint.
Where possible, templates will be built from all available commissioning data before the start of year one and used to generate alerts during year one.
How extensive these templates are at the start of full survey operations will be influenced by the overall success of commissioning.
During year 1, templates will be built progressively from data obtained during year one (e.g., on a monthly timescale), and used to generate alerts during year one, either instead of, or in addition to using commissioning data to build templates.
