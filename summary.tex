\section{Summary}

Rubin Observatory is putting in place a dedicated {\it Early Science Program} to ensure high-impact science during the first year of operation of the \lsst.
This Early Science Program is motivated by the fact that in the current baseline there will be no science-ready data products released before \drone. 

One of the four science pillars of LSST is time-domain astronomy, which is enabled by alerts on LSST detections of transient, variable, and/or moving objects. 
Alerts are the only data product that will be immediately available (within 60 seconds of image readout) and publicly shareable, i.e not subject to a proprietary period \citep{LSE-163},  \citep{RDO-013}. 
The worldwide community is actively preparing to process the LSST alert stream and use it to generate groundbreaking scientific results. Additionally, for many science goals, time-sensitive follow-up observations after discovery are crucial to take full advantage of the Rubin data. 

A key component of the \esp is the capability to build {\it incremental} templates from on-sky imaging as it becomes available during commissioning and the early phases of the survey. 
Such templates will be built periodically as images accumulate to allow for partial alert generation over an incomplete sky footprint.
Where possible, templates will be built from all commissioning data before the start of year one and used to generate alerts during year one.
How extensive these templates are at the start of full survey operations will be influenced on the overall success of commissioning.
During year 1, templates will be built progressively from data obtained during year one (e.g., on a monthly timescale), and used to generate alerts during year one, either instead of, or in addition to using commissioning data to build templates.

The data acquired during commissioning will be released to the Rubin data rights community via Data Previews and will include data products for both static-sky science and time-domain science.