\section{Summary}

Science from the \rolsst will be enabled by two categories of data products: Prompt data products and Data Release data products (\citep{LPM-231}, \citep{LSE-163}). 
Data Release data products result from a coherent processing of the entire science dataset to date, and will be released as a series of 11 approximately annual data releases over the 10-year period of the survey (\citep{RDO-011}).
Prompt data products are generated continuously every observing night by the real-time Prompt Processing pipelines using Difference Image Analysis (DIA). 
Five-sigma detections will be reported within 60s of image readout in the form of a world-public stream of {\it Alerts}, distributed to Community Brokers for further analysis and follow-up by the Rubin science community.  
Alerts of transient, variable, and/or moving objects are the only data product that will be immediately available and publicly shareable, i.e not subject to a proprietary period.
The worldwide community is actively preparing to process the LSST alert stream and use it to generate groundbreaking scientific results. 
 
Difference Image Analysis (DIA) requires transient-free coadded template images of the sky.
Templates are generated as part of the annual Data Release Processing (DRP), and thus in the current survey baseline, full volume, full fidelity Alert production can only begin in year 2 of operations. 

Some expectations of early science have been built around the verification phase of commissioning,  which is designed to include at least two months of sustained observing. 
Templates can be built from commissioning data obtained with the LSST camera and used to generate alerts in year 1 for the corresponding field and filter. 



The early science program will be steadily evolved throughout pre-operations as the SIT-COM activities take place, adapting to the state of the system and with an eye on verification and validation data sets being produced in commissioning.

LSST is expected to produce about 10 million alerts per night in {\it steady state} operations. 

A key component of the ESP is the design a template acquisition strategy during both commissioning and the early phases of the survey, to increase the 
is committed to	

Some expectation of early science has always been built around the verification phase of commissioning which is designed to have at least two months of sustained observing (see above). 

\ro is committed to ensuring a high-impact Early Science Program



Science during the first year of Rubin Observatory / LSST operations is a high priority. 
Prior to \drone, scientific investigations may be undertaken with data processed in Alert Production and Solar System Processing as well as \dpvs of the commissioning data.
Alerts of transient, variable, and/or moving objects are the only data product that will be immediately available (within 60 seconds of image readout) and publicly shareable, i.e not subject to a proprietary period. 
Alert Production requires high-quality template images of the sky, and thus full volume and full fidelity alerts will not be available until after \drone.
However, \ro is planning a process of incremental template generation to maximize the time-domain and solar system science achievable in the first year of operations.
The worldwide community is actively preparing to process the LSST alert stream and use it to generate groundbreaking scientific results. 
This note describes the \ro plan for ensuring \es. 
It is a living document that will evolve over the course of the remainder of the construction project and up until \drone. 