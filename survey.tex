% Implication of early science on the full survey cadence
% Responsible : Zeljko
\section{Survey Cadence}

\TODO{Phil, for review by Bob and Zeljko}{We can now make an iteration on this section to claim minimal impact on the LSST.}

% Early Science observations should align as closely as possible with main survey and ultimate long-term science goals
% Details on the implications for the Survey Cadence of the \esp will be added in future as the outcome of commissioning becomes clear.

The early science observations can be separated into two kinds: commissioning observations, and survey observations taken during the first 6 months of the LSST.
Only the latter---the ``early survey observations`` will be included in the LSST data processing.
Optimizing the early survey observing schedule for incremental template generation and alert production in the first 6 months will mean that the time sampling in about half the LSST survey region in Year 1 may look somewhat different to that in subsequent years.
However, the Year 1 coadd depth should, to first order, be as uniform as any other year in the LSST (although there could also be some second order effects).
The Survey Cadence Optimization Committee (SCOC) will work with the Rubin Survey Scheduling Team to study the impacts of optimizing the early survey observations for early science, and also take input from the LSST science community on further, small adjustments to the early survey cadence so as to enable particular aspects of early science.
