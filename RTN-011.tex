\documentclass[DM,authoryear,toc]{lsstdoc}
% lsstdoc documentation: https://lsst-texmf.lsst.io/lsstdoc.html
\input{meta}

% Package imports go here.

% Local commands go here.

%If you want glossaries
%\input{aglossary.tex}
%\makeglossaries

\title{Plans for Early Science}

% Optional subtitle
% \setDocSubtitle{A subtitle}

\author{%
Leanne Guy
}

\setDocRef{RTN-011}
\setDocUpstreamLocation{\url{https://github.com/rubin-observatory/rtn-011}}

\date{\vcsDate}

% Optional: name of the document's curator
% \setDocCurator{The Curator of this Document}

\setDocAbstract{%
Science during year one is a priority for Rubin Observatory operations. This document provides the plans for enabling early science.  It is a living document that will evolve over the course of the pre-operations period. 
}

% Change history defined here.
% Order: oldest first.
% Fields: VERSION, DATE, DESCRIPTION, OWNER NAME.
% See LPM-51 for version number policy.
\setDocChangeRecord{%
  \addtohist{1}{YYYY-MM-DD}{Unreleased.}{Leanne Guy}
}


\begin{document}

% Create the title page.
\maketitle
% Frequently for a technote we do not want a title page  uncomment this to remove the title page and changelog.
% use \mkshorttitle to remove the extra pages

% ADD CONTENT HERE
% You can also use the \input command to include several content files.

\appendix
% Include all the relevant bib files.
% https://lsst-texmf.lsst.io/lsstdoc.html#bibliographies
\section{References} \label{sec:bib}
\renewcommand{\refname}{} % Suppress default Bibliography section
\bibliography{local,lsst,lsst-dm,refs_ads,refs,books}

% Make sure lsst-texmf/bin/generateAcronyms.py is in your path
\section{Acronyms} \label{sec:acronyms}
\addtocounter{table}{-1}
\begin{longtable}{p{0.145\textwidth}p{0.8\textwidth}}\hline
\textbf{Acronym} & \textbf{Description}  \\\hline

<<<<<<< HEAD
DM & Data Management \\\hline
RTN & Rubin Technical Note \\\hline
=======
 &  \\\hline
B & Byte (8 bit) \\\hline
DM & Data Management \\\hline
DM-SST & DM System Science Team \\\hline
DMS & Data Management Subsystem \\\hline
DMS-REQ & Data Management System Requirements prefix \\\hline
DMTN & DM Technical Note \\\hline
DR1 & Data Release 1 \\\hline
DRP & Data Release Processing \\\hline
LCR & LSST Change Request \\\hline
LSST & Legacy Survey of Space and Time (formerly Large Synoptic Survey Telescope) \\\hline
MAF & Metrics Ananlysis Framework \\\hline
OSS & Observatory System Specifications; LSE-30 \\\hline
PP & Prompt Processing \\\hline
PST & Project Science Team \\\hline
RTN & Rubin Technical Note \\\hline
SIT & System Integration, Test \\\hline
SIT-COM & System Integration, Test, and Commissioning \\\hline
SST & Subsystem Science Team \\\hline
SV & Science Verification \\\hline
TOO & Target Of Opportunity \\\hline
>>>>>>> 1e5fe2e230228a6b53728a184090b91c5b8a7052
\end{longtable}

% If you want glossary uncomment below -- comment out the two lines above
%\printglossaries





\end{document}
