\documentclass[DM,authoryear,toc]{lsstdoc}
% lsstdoc documentation: https://lsst-texmf.lsst.io/lsstdoc.html
\input{meta}

% Package imports go here.
\usepackage{xspace}
\usepackage{hyperref}

% Local commands go here.
\newcommand{\es}{Early Science\,}\xspace
\newcommand{\sv}{Science Verification\,}\xspace
\newcommand{\dpv}{Data Preview\,}\xspace
\newcommand{\dpvs}{Data Previews\,}\xspace
\newcommand{\dpzero}{Data Preview 0 (DP0)\,}\xspace
\newcommand{\dpone}{Data Preview 1 (DP1)\,}\xspace
\newcommand{\dptwo}{Data Preview 2 (DP2)\,}\xspace
\newcommand{\drone}{Data Release 1 (DR1)\,}\xspace
\newcommand{\tvssc}{Transients and Variable Stars Science Collaboration (tvssc)\,}\xspace
\newcommand{\sssc}{Solar System Science Collaboration (sssc)\,}\xspace
\newcommand{\diffim}{Difference Imaging\,}\xspace
\newcommand{\drp}{Data Release Production (DRP)\,}\xspace
\newcommand{\ap}{Alert Production (AP)\,}\xspace
\newcommand{\ro}{Rubin Observatory\,}\xspace
\newcommand{\rolsst}{Rubin Observatory Legacy Survey of Space and Time (LSST)}\xspace
\newcommand{\vrolsst}{Vera C. Rubin Observatory Legacy Survey of Space and Time (LSST)\xspace}
\newcommand{\esp}{Early Science Program (ESP)\,}\xspace 

\setcounter{tocdepth}{4}
%If you want glossaries
%\input{aglossary.tex}
%\makeglossaries

\title{Rubin Observatory Early Science Program}

% Optional subtitle
% \setDocSubtitle{A subtitle}

\author{%
Leanne~P.~Guy, Eric~Bellm, Bob~Blum, \v{Z}eljko~Ivezi\'{c}, Michael~Strauss}

\setDocRef{RTN-011}
\setDocUpstreamLocation{\url{https://github.com/rubin-observatory/rtn-011}}

\date{\vcsDate}

% Optional: name of the document's curator
\setDocCurator{Leanne Guy}

% Leanne 
\setDocAbstract{%
High-impact scientific results during the first year of \rolsst operations are a priority. 
This document provides the plans for a dedicated Early Science Program designed to realize that goal. 
It describes the strategy for obtaining observations during commissioning and in the early phases of the survey, and how they will align to the main survey and ultimate long-term science goals. 
It outlines how the Rubin operations team will work with the science community to optimise the various aspects of the Early Science Program, driven by opportunities for high-impact early science.
This is a living document; both it and the Early Science Program will continue to evolve over the course of commissioning and pre-operations in response to the state of the as-built system and to community input.
}


% Change history defined here.
% Order: oldest first.
% Fields: VERSION, DATE, DESCRIPTION, OWNER NAME.
% See LPM-51 for version number policy.
\setDocChangeRecord{%
  \addtohist{1}{2020-10-30}{First draft}{Leanne Guy}
  \addtohist{2}{2020-12-16}{Draft 1.1}{Bob Blum}
  \addtohist{3}{2021-10-08}{Rework structure}{Leanne Guy}
  \addtohist{4}{2021-10-21}{Add timeline}{Leanne Guy}
  \addtohist{5}{2021-11-05}{Edits throughout}{Eric Bellm}
  }


\graphicspath{{./}{figures/}}
\setcounter{tocdepth}{3}

%%%%%% Begin %%%%%%
\begin{document}

% Create the title page.
\maketitle
% Frequently for a technote we do not want a title page  uncomment this to remove the title page and changelog.
% use \mkshorttitle to remove the extra pages

% Leanne
\section{Summary}

Rubin Observatory is putting in place a dedicated {\it Early Science Program} to ensure high-impact science prior to the release of the first year of \lsst data.
The Rubin Operations team is carrying out a series of ``Data Previews,'' in which LSST pre-cursor data products are prepared, released to the LSST data rights community, and supported.
The goals of the Data Previews are to 1) enable the Operations teams to develop operational capability prior to the start of the LSST, and 2) support the members of the LSST science community as they develop their LSST analyses.
The first Data Preview era, DP0, is already underway: DP0.1 and DP0.2 contain Rubin-processed image and catalog data products derived from simulated LSST images that were generated by the LSST Dark Energy Science Collaboration as the DC2 Virtual Sky Survey \citep{2021ApJS..253...31L}.
A third phase, DP0.3, will focus on catalog-level simulated Solar System science products and is scheduled for release in 2023.
Subsequent Data Previews DP1 and DP2 will contain LSST-like data products generated from the commissioning data taken with LSSTCam, respectively.
The data taken during the first 6 months of the LSST survey will be processed and released as Data Release 1.
DP1, DP2, and DR1 will all include data products for both static-sky science and time-domain science.

Time-domain astronomy is a key component of LSST's four science pillars and is enabled by alerts on LSST detections of transient, variable, and/or moving objects.
Alerts are the only data product that will be immediately available (within 60 seconds of image readout) and publicly shareable, i.e. not subject to a proprietary period \citep{LSE-163},  \citep{RDO-013}.
The worldwide community is actively preparing to process the LSST alert stream and use it to generate groundbreaking scientific results. Additionally, for many science goals, time-sensitive follow-up observations after discovery are crucial to take full advantage of the Rubin data.

A key component of the \esp is the capability to build {\it incremental} templates from on-sky imaging as it becomes available during commissioning and the early phases of the survey.
Such templates will be built periodically as images accumulate to allow for partial alert generation over an incomplete survey footprint.
Where possible, templates will be built from all available commissioning data before the start of year one and used to generate alerts during year one.
How extensive these templates are at the start of full survey operations will be influenced by the overall success of commissioning.
During year 1, templates will be built progressively from data obtained during year one (e.g., on a monthly timescale), and used to generate alerts during year one, either instead of, or in addition to using commissioning data to build templates.

\clearpage

% ADD CONTENT HERE
% Leanne
% Merge summary and introduction 
\section{Rubin Early Science Program}

Community expectations for early science with Rubin are high due to the transformative nature of the LSST data and the densely-sampled observations planned during the commissioning period.
Rubin Observatory's \emph{Early Science Program} is designed to provide Rubin data rights holders with access to the data products and services necessary to produce high-impact early science during time between commissioning through, and including, the first data release, Data Release 1 (DR1). 

\subsection{Definition of Early Science}  \label{ssec:defn}
Early Science is defined as any science enabled by Rubin for its community through and including the first LSST Data Release, DR1.

\subsection{Elements of the Early Science Program}

The Early Science Program consists of the following elements:
\begin{itemize}
	\item A series of three \textbf{Data Previews (DP)}, DP0, DP1 and DP2,  based on simulated LSST-like data and reprocessed data taken during the Rubin Observatory commissioning period with the LSST Science Camera (LSSTCam). 
	\item A world-public \textbf{stream of alerts} from transient, variable, and moving sources that will be scaled up continuously during commissioning and the first year of the survey. 
	\item  \textbf{Template generation}, both prior to the start of regular survey operations based on data collected during the commissioning period with LSSTCam, and incrementally during the first year of regular survey operations  to maximize the number of templates available for Alert Production in year 1. 
	\item \textbf{LSST Data Release 1 (DR1)}, which will be based on the Data Release Processing (DRP) of the first six months of LSST data.
\end{itemize}


\subsection{Early Science scenarios } \label{ssec:scenarios}

Recent planning on the construction project has led to a reduced amount of on-sky time in commissioning, including a reduction in the time dedicated to final science validation of the as-built system compared to earlier draft plans.
The total amount science validation time currently planned during commissioning is 8 weeks.
As Rubin construction moves through the challenging phase of System Integration, Test and Commissioning (SIT-Com), on--sky time could be further reduced.

The Operations team is tracking the progress of the commissioning activities as they relate to Early Science opportunities to ensure that the community has timely access to science-ready data products while the survey begins its relentless coverage of the sky leading to DR1.
We broadly envisage two possible scenarios emerging from the commissioning phase of the construction project: 

\begin{itemize}
\item \textbf{Scenario A}:
The full commissioning plan comprising system optimization and science validation is successfully executed as planned. 
Rubin Operations then carries out an Operations Rehearsal and Operations Readiness Review (ORR) to effectively conduct a \textit{full dress rehearsal} of science operations and demonstrate the readiness of the Operations team to execute the 10-year survey. 
Science-grade data collected during the commissioning System Optimization period and subsequent Science Validation Surveys, \S~\ref{ssec:commissioning}, is reprocessed to produce the final Data Preview, DP2, which will be released 6 months following the completion of the Science Validation Surveys.

\item \textbf{Scenario B}:
On-sky time in commissioning is reduced as the construction work  draws to an end, resulting in the SV surveys not being completed prior to the end of the construction phase.
The Operations team would spend up to 3 months prior to commencing the 10-year LSST survey completing any remaining SV Survey observations.
As per Scenario A, data collected during commissioning and the SV Surveys is reprocessed to produce DP2 and an Operations Readiness Review carried out to demonstrate readiness to execute the 10-year survey. 
\end{itemize}

A key point to note is that the DP2 data products will be the same irrespective of which scenario materializes.
Only the timing of the release of DP2 and the start of the 10-year survey are different between the two scenarios.
In both scenarios it is assumed that the Rubin Construction project delivers an integrated system that can capture, transfer and process science-grade data at the time Operations begins.
Both scenarios will include alert generation of some type, with the major distinction being the relative availability of templates in time, sky position, and filter.

These two scenarios presented are current as of December 2022, however are subject to change commissioning program emerges and is executed.
At some future point, a single option will be adopted and executed, and at that time, the details will be more fully specified.



% Leanne 
\section{Roadmap and Timeline} \label{sec:timeline}

Table~\ref{tab:milestones} provides a list of key milestones for Rubin Operations and the Early Science Program, as of January 2023.
It will continue to be updated as Rubin Construction and the Early Science Program progress. 
The date ranges are derived from the Rubin ``Celebratory Milestones'', which are  published monthly on the Rubin Project website\footnote{\url{https://www.lsst.org/about/project-status}}. 

\begin{table}[htb]
\label{tab:milestones}
\includegraphics[width=\linewidth]{figures/DPR-milestones}
\caption{Top milestones for the Early Science Program, as of January 2023.}
\end{table}

Milestone dates are given as min-max ranges to indicate the associated uncertainty. 
Typically the near date corresponds to the current Project forecast, plus any additional operational uncertainty.
The late date corresponds (approximately) to the current Project ``late date'' plus any additional operational uncertainty and cannot be surpassed without the Project re-baselining its schedule.
An intermediate (typically mid-range) date is used by the Rubin Operations teams for planning purposes. 

The LSST survey will start shortly after the completion of the SV surveys, currently expected to be sometime between late 2024 and early 2025.
The timing of the Commissioning observations is somewhat less uncertain and the timing of the release of those data to the community can be projected to within a few months at the time of writing.

Table \ref{tab:timeline} shows the nominal  dates for the various elements of the Early Science Program. 
The next key milestone in the Early Science Program is the release of DP0.3. 
Scheduled for mid-2023, DP0.3 will be the last in the DP0 series, which is based on simulated LSST-like data. 
The late dates for the DP2 and DR1 milestones allow for the possibility that the Project completes within its late date, but in doing so reduces the amount of on-sky LSSTCam commissioning time.
In this eventuality, the operations team would spend up to 3 months prior to commencing the 10-year L SST survey completing any remaining SV Survey observations , such that DP2 could be realized as planned.

\begin{table}[htb]
\label{tab:timeline}
\includegraphics[width=\linewidth]{figures/DPR-timeline}
\caption{Nominal dates for the various elements of the Early Science Program, as of January 2023.}
\end{table}

% Factor this statement to the begining?
Tables ~\ref{tab:milestones} and ~\ref{tab:timeline} will continue to be refined and updated in future version of this documents as the Early Science Program progresses 


% Leanne 
\section{Science Drivers} \label{sec:science}

The various different science drivers outlined in \ref{sec:science} naturally lead to different priorities for template generations, e.g. solar system science prefers templates to be generated in the NES and Milky Way science would to would prefer templates for the galactic plane to optimise alert production in these areas in early operations. Other science will prefer templates in a number of filters to enable .. rather that larger area. 

\subsection{Time Domain}

The \tvssc reviewed the opportunities for \es for non time-critical and  time-critical science cases in \cite{Hambleton_2020} and \cite{Street_2020} respectively. 

\subsection{Solar System}

The \sssc reviewed opportunites for \es in \citeds{2020arXiv201005926L}. 
LSST is predicted to discover $\approx$ 6 million solar system planetesimals, providing in total over a billion photometric and astrometric measurements in 6 broad-band filters. 

\subsection{Static Science}
The baseline static science data sets will flow from \sv surveys carried out during commissioning. 

\subsection{Target of Opportunity}
Rubin Observatory will be prepared to take advantage of Targets of Opportunties (TOO) in the first year of operations (and hopefully SIT-COM). 
\citedsp{RTN-008} describes potential data processing scenarios for TOO observations in the early operations era.

% Eric

%% Leanne - restructure to present the prompt data products better 
% Add in from data preview section: 
%
%During routine LSST operations, prompt image data products will be made available 80 hours following camera readout.
%They include raw images, processed single visit images (PVIs), difference images, and template images.
%Access to unvetted PVIs and difference images in the first 6 months of the LSST is still to be decided.


\section{Alert Production in Commissioning and Early Operations}
\label{sec:pp}

The \DPDD{} summarizes the pipelines that will be used during Prompt Processing to produce alerts as well as other prompt data products, including Solar System Processing.
Both Alert Production and Solar System Processing depend on the existence of template images.
During steady-state operations, these templates will be constructed during the annual Data Releases and will be built from the best available subset of images taken.
To enable alert production to proceed during commissioning and early operations, it is necessary build templates incrementally as data become available, as recommended by the study described in \citeds{DMTN-107}.
Because we have a smaller set of input images to choose from and uncertain knowledge about future observations, incremental template generation necessarily must balance the trade-off of earlier template availability against template quality and spatial completeness.
Validation will be required to determine when to build incremental templates to maximize the net throughput of Early Science.
Nevertheless, our goal is to enable Alert Generation to begin over at least a subset of the survey area as soon as the data are scientifically useful.

Scientifically it is important that once a template is constructed for a given region of sky, it is used exclusively until it can be updated in the next Data Release.
Repeated changes to the template make it extremely difficult to construct usable lightcurves for objects from individual difference image sources: transient objects such as supernovae will be contaminated by changing flux levels from the evolving template, and variable objects such as variable stars and AGN will require repeated corrections for different template flux levels as well.

During commissioning templates will be generated incrementally over the maximal sky area supported by the available observations.
By the end of the commissioning period, coadd templates for use in difference imaging will only be available for $\approx$ 10\% of the sky.
Generating templates over a wide area is not an explicit goal of commissioning;  however, where possible, if commissioning observations are agnostic to pointing and filter, we would endeavour to choose a pointing and filter that maximizes building templates to enable early science.

Rubin aims to scale up alert production during commissioning with the aim of beginning routine Alert Production as soon as is feasible following System First Light  (\S~\ref{sec:timeline}).
\citeds{RTN-061} describes the criteria for sending the first Rubin alerts.
Once begun, Alert Production will then proceed continuously into the full LSST survey.
Alerts generated during commissioning may be produced with higher latency, and access to images and the PPDB may not be immediately available.

Table~\ref{tab:prompt-data-products} lists the various alert and prompt processing data products currently planned at each phase of alert production during commissioning and the first two years of the LSST survey. 
Phase 1 covers the commissioning period and phase 2 covers early survey operations.  
During routine LSST operations, prompt image data products, including  raw images, processed single visit images (PVIs), difference images, and template images, will be made available no earlier than 80 hours following camera readout. 
During the first 6 months of the LSST,  prompt PVIs and difference images may be released with higher latency as Rubin continues to understand data quality and scale up services.


\begin{table}
\centering
\fontsize{7}{11}\selectfont 
\setlength{\tabcolsep}{6pt} % Default value: 6pt
{\renewcommand{\arraystretch}{1.3}
    \begin{tabular}{|p{0.31\linewidth} | p{0.32\linewidth}  | p{0.32\linewidth}|}
    \hline
    \multicolumn{3}{|l|}{{\fontsize{9}{12}\selectfont \color{RubinDarkTeal}\textbf{Rubin Early Alerts \& Prompt Products}}}  \\\hline\hline
\multirow{1}{*} {}  & 
        \tiny  \makecell{Phase 1: 3 -- 16 weeks post System First Light}  & 
        \tiny   \makecell{Phase 2: 18 -- 17 weeks post System First Light} \\[5pt] \cline{2-3}        
        {\parbox{0.5\linewidth}{\vspace{0.6cm} \textbf{Data Product}}}  &   
        { \makecell{ \textbf{Commissioning } \\ \textbf{(System Optimization \& SV Surveys} }}  & 
        {\makecell{\textbf{Year 1 Survey} \\ \textbf{Operations} }} 
         \\[10pt] \cline{2-3} \hline\hline

\textbf{PP Processed Visit Images}     & Commissioning of the PP image differencing and incremental template building will begin shortly after System First Light. Prompt image release is embargoed during commissioning.  &   Access to unvetted processed visit images as prompt products in the first 6 months of the LSST is still TBD. PP PVIs available after DP2.     \\  \arrayrulecolor{gray}\hline
\textbf{PP Difference Images}     & Difference imaging will be somewhat limited, since the image template sky coverage will be sparse. Prompt image release is embargoed during commissioning.  &     Difference imaging will steadily increase as incremental template building increases the templates available. Prompt access to unvetted PP difference images is still TBD. PP diff images available after DP2.    \\\hline
\textbf{PP Catalogs}    &   PPDB likely unavailable for query. &  PPDB available for query. \\ 
 (DIASources, DIAObjects, DIAForced Sources)  & & \\\hline
\textbf{PP SSP Catalogs}   &   Alert volume and latency will improve throughout the commissioning period. Aiming for "near-live" brokered Alert stream by the end of LSSTCam SV.  &   Difference imaging will steadily increase as incremental template building increases the templates available. Prompt access to unvetted PP difference images is still TBD. PP diff images available after DP2. \\  \hline
\textbf{PP SSP Catalogs}   &   Measurements of known SSObjects sent to the MPC whenever difference images are available. Searches for new SSObjects performed if appropriately-cadenced data is present. SSP Catalogs likely unavailable for query in the PPDB. &   Standard SSP Daily Data Products produced from difference images as they are available and reported to the MPC. SSP catalogs available for query in the PPDB.  \\  \hline

\arrayrulecolor{black}\hline
\end{tabular}}
\caption{Summary of prompt data products expected during commissioning and year 1survey observations..}
\label{tab:prompt-data-products}
\end{table}

%\begin{table}[ht]
%\centering
%\includegraphics[width=0.9\linewidth]{figures/Prompt-products}
%\caption{Summary of prompt data products expected during commissioning and year 1 survey observations.}
%\label{tab:prompt-data-products}
%\end{table}



% Leanne 
\section{Rubin Observatory Commissioning}
\label{sec:commissioning}

\subsection{Commissioning Schedule}
\label{ssec:commissioning-schedule}

Figure~\ref{fig:commissioning-es-schedule} shows the detailed schedule of commissioning and early science activities relative to System First Light, as of \currentdate.
ComCam First Photon was successfully achieved on 24 October 2024.
Rubin First Photon, with LSSTCam, is currently expected on 15 April 2025 and System First Light in  July 2025 (\S~\ref{sec:timeline}). 
The total amount of science validation time currently planned with LSSTCam is about 8 weeks.  
LSST data taking is expected to start 4-6 months after System First Light depending on construction schedule uncertainty and Rubin Operations readiness to start the survey. 

\begin{figure}[htb]
\centering
\includegraphics[width=0.98\linewidth]{figures/rubinobs_on-sky_commissioning_and_early_science.pdf}
\caption{Detailed schedule of commissioning  and early science activities relative to System First Light, as of \currentdate.}
\label{fig:commissioning-es-schedule}
\vspace{0.1cm}
\end{figure}

Rubin Observatory carried out an on-sky commissioning campaign with ComCam from 24 October 2024  through 15 December 2024. 
The campaign was very successful, accomplishing many goals including the optical alignment of the telescope and the delivery of science-grade images.
The median delivered image quality  for commanded in-focus images collected during the campaign, quantified in terms of the PSF FWHM, was $\approx$1.1 arcseconds. 
The best images have delivered PSF FWHM of $\approx$0.7 arcseconds.
A full report on the ComCam on-sky commissioning campaign is available at \citeds{sitcomtn-149}.

The project schedule will continue to evolve as the remaining subcomponents are delivered. 

\subsection{Commissioning Milestones}
\label{ssec:commissioning-milestones}

Commissioning work is being planned around three major milestones, \textit{ComCam First Photon}, \textit{LSSTCam First Photon} and \textit{System First Light}. 

\textbf{ComCam First Photon}: The first image of the night sky produced by photons passing through the Rubin optical system and detected by the Commissioning Camera (ComCam). 
This milestone was achieved on 24 October 2024. 

\textbf {LSSTCam First Photon}: The first image of the night sky produced by photons passing through the Rubin optical system and detected by the LSST Science Camera (LSSTCam).

\textbf {System First Light}: Defined as the point at which we can routinely acquire science-grade imaging across the LSSTCam full focal plane and have a well understood technical path towards meeting the Construction Completeness criteria   \citeds{sitcomtn-061}.
Also referred to as \textbf{First Light}. 

LSSTCam First Photon occurs following the successful completion of system integration. 
There are no quality criteria applied to achieving  neither the ComCam nor LSSTCam First Photon milestones. 
System First Light  marks the end of the  on-sky engineering phase and the start of the System Optimization and Science Validation phases of commissioning.
The period between ComCam First Photon and System First Light will focus on fine tuning the system including optical alignment and improving the image quality, collecting calibration data, and carrying out \textit{First Look} science programs. 

For a detailed description of all the commissioning milestones and the most current dates, see \citeds{dmtn-232}.


\subsection{Commissioning Observations}
\label{ssec:commissioning-observations}

Figure~\ref{fig:commissioning} shows the high level plan for the Rubin commissioning observations. 
Commissioning data collection is planned to take place in phases.
The \textit{On-Sky Engineering} phase may be carried out with either ComCam and/or LSSTCam, depending on future re-optimization of the sequence of integration activities (\S~\ref{ssec:commissioning-schedule})
During the \textit{System Optimization} phase,  a set of observations designed to help optimize the system will be taken during the System Optimization phase before the Science Validation Surveys are carried out. 
The SV Surveys are designed to support scientific analyses that validate the system's performance, and allow Rubin to demonstrate operations readiness \citeds{SITCOMTN-005}.

\begin{figure}[htb]
\centering
\includegraphics[width=0.95\linewidth]{figures/commissioning-plan}
\caption{Outline plan for the collection of commissioning data, as of \currentdate.}
\label{fig:commissioning}
\end{figure}

Figure~\ref{fig:commissioning} also indicates a number of planned key components of the System Optimization and SV phases.
These include a LSST wide-fast-deep (WFD) 1-2 year equivalent depth ``pilot'' survey.
Field selection will be carried out by the Commissioning Team, taking into account a wide variety of constraints as well as a ``menu'' of science opportunities to which the LSST Science Community has contributed.
Details of the plans for commissioning observations will be made available as those plans converge, in this technote and other documents as cited.

% Leanne
% Early Science Data Products
% Responsible: Leanne Guy
\section{Early Science Data Products} 
\label{sec:data}

\subsection{Prompt data products}
Describe the alert data products

\subsection{Data Release data products}
Images and catalogs from the DRP of the commissioning data will be made available to  data rights holders via the access mechanisms described in \ref{ssec:dataaccess}.

\subsection{Access to \es data products}\label{ssec:dataaccess}
Describe how the \es  data products will be accessed by the community. 
Describe the community brokers, the science platforms and how the \dpvs play a role in providing access to \es data products. 
Explain that the same data rights policy applies to \es data products. \cite{RDO-013}.




% Zeljko 
%% Implication of early science on the full survey cadence
% Responsible : Zeljko
\section{Survey Cadence}

Details on implications for the Survey Cadence of the \esp will be added in future. 



% Amanda
%\section{Community Engagement}

Rubin Observatory will work closely with the Survey Cadence Optimization Committee (SCOC) and Community on the detailed design of the Early Science Program. 

\subsection{Survey Cadence Optimization Committee}
The Rubin Survey Cadence Optimization Committee (SCOC)\footnote{See \url{https://www.lsst.org/content/charge-survey-cadence-optimization-committee-scoc}} is an advisory committee to the Rubin Observatory Operations Director consisting of 10 members drawn almost entirely from the science community.
Convened in 2020, the SCOC will be a standing committee throughout the lifetime of Rubin Observatory operations and will be involved in all aspects of the development of the Early Science Program. 

The SCOC will work with Rubin and the Community to make specific recommendations for early science observations in terms of, for example, the prioritization of sky coverage, filters, and other specific choices. 
Recommendations will take into account the plans for commissioning and the realized performance of the telescope and software, and should align as closely as possible with those of the main survey and ultimate long-term science goals. 
Optimizing the LSST Year 1 observing schedule for early science may mean that the time sampling looks somewhat different to that in subsequent years. 

Work on recommendations for Early Science observations have been deferred until 2023 after the release of the SCOC Phase 2 recommendations.
The SCOC will solicit input from the community on the specific observing strategy in year 1 to optimize early science. 
Several science collaborations have already been pro-active in providing input,  both the community forum and as research notes (\citep{2020arXiv201005926L}, \citep{Hambleton_2020}, \citep{Street_2020}).


\subsection{Community Forum}

The Rubin Observatory Community Platform has a dedicated category for Early Science\footnote{See \url{https://community.lsst.org/t/about-the-early-science-category/5775}}, where community members are encouraged to open discussions on the topic of early science. 
Community feedback on the Early Science data products is welcomed and will help the Rubin to improve data products and services. 


\appendix
% Include all the relevant bib files.
% https://lsst-texmf.lsst.io/lsstdoc.html#bibliographies
\section{References} \label{sec:bib}
\renewcommand{\refname}{} % Suppress default Bibliography section
\bibliography{local,lsst,lsst-dm,refs_ads,refs,books}

% Make sure lsst-texmf/bin/generateAcronyms.py is in your path
\section{Acronyms} \label{sec:acronyms}
\addtocounter{table}{-1}
\begin{longtable}{p{0.145\textwidth}p{0.8\textwidth}}\hline
\textbf{Acronym} & \textbf{Description}  \\\hline

<<<<<<< HEAD
DM & Data Management \\\hline
RTN & Rubin Technical Note \\\hline
=======
 &  \\\hline
B & Byte (8 bit) \\\hline
DM & Data Management \\\hline
DM-SST & DM System Science Team \\\hline
DMS & Data Management Subsystem \\\hline
DMS-REQ & Data Management System Requirements prefix \\\hline
DMTN & DM Technical Note \\\hline
DR1 & Data Release 1 \\\hline
DRP & Data Release Processing \\\hline
LCR & LSST Change Request \\\hline
LSST & Legacy Survey of Space and Time (formerly Large Synoptic Survey Telescope) \\\hline
MAF & Metrics Ananlysis Framework \\\hline
OSS & Observatory System Specifications; LSE-30 \\\hline
PP & Prompt Processing \\\hline
PST & Project Science Team \\\hline
RTN & Rubin Technical Note \\\hline
SIT & System Integration, Test \\\hline
SIT-COM & System Integration, Test, and Commissioning \\\hline
SST & Subsystem Science Team \\\hline
SV & Science Verification \\\hline
TOO & Target Of Opportunity \\\hline
>>>>>>> 1e5fe2e230228a6b53728a184090b91c5b8a7052
\end{longtable}

% If you want glossary uncomment below -- comment out the two lines above
%\printglossaries

\end{document}
