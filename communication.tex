\section{Communication}

\subsection{Community Platform}

A category dedicated to Early Science has been 
Rubin community members can open discussions on the topic of early science on the \href{https://community.lsst.org/t/about-the-early-science-category/5775}{Rubin Community Forum Early Science category}

Describe here the process by which the community will be consulted and decisions will be made about the early science programme. 

Points to address:
\begin{itemize}
\item SCOC - see Zeljko SciCollab talk 20211027
\item Use of the community platform for engaging the community to provide input,
\item Process by which we officially solicit  input from the community on preferences for early science, e.g number of filters, vs area vs pointings vs .... Each different science has a different preference. 
\item Decision making criteria 
\item Decision making body
\item Timeline for making collecting input and making a decision
\end{itemize}

\subsection{Relevant Publications}

Several science collaborations have already been pro-active in providing input on considerations for template generation in year one in year one. 
{\texbf  Impact of Rubin Observatory LSST Template Acquisition Strategies on Early Science from the Transients and Variable Stars Science Collaboration: Time-critical Science Cases.} \citeds{Hambleton_2020}.

{\texbf  Impact of Rubin Observatory LSST Template Acquisition Strategies on Early Science from the Transients and Variable Stars Science Collaboration: Time-critical Science Cases,} \citeds{Street_2020}.

{\texbf  Opportunities for High Impact Solar System Science During Year 1 of the Legacy Survey of Space
and Time (LSST)}, \citeds{2020arXiv201005926L} .`
