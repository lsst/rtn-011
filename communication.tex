\section{Community}
\label{sec:community}
Rubin Observatory will work closely with the Survey Cadence Optimization Committee (SCOC) and Community on the detailed design of the Early Science Program. 

\subsection{Survey Cadence Optimization Committee}
The Rubin Survey Cadence Optimization Committee (SCOC)\footnote{See \url{https://www.lsst.org/content/charge-survey-cadence-optimization-committee-scoc}} is an advisory committee to the Rubin Observatory Operations Director consisting of 10 members drawn almost entirely from the science community.
Convened in 2020, the SCOC will be a standing committee throughout the lifetime of Rubin Observatory operations and will be involved in all aspects of the development of the Early Science Program. 

The SCOC will work with the Rubin Operations team and the Community to establish the best strategy for Early Science, including making specific recommendations in terms of, for example, the prioritization of sky coverage, filters, and other specific choices. 
Recommendations will take into account the plans for commissioning and the realized performance of the telescope and software, and should align as closely as possible with those of the main survey and ultimate long-term science goals. 
Optimizing the LSST Year 1 observing schedule for early science may mean that the time sampling looks somewhat different to that in subsequent years. 

The SCOC has published its Phase 1, 2  and 3 survey cadence recommendations in \citeds{PSTN-053}, \citeds{PSTN-055} and \citeds{PSTN-056} respectively. 
The SCOC will solicit input from the community on the specific observing strategy in year 1 to optimize early science. 
Several science collaborations have already been proactive in providing input,  both the community forum and as research notes (\citep{2020arXiv201005926L},~\citep{Hambleton_2020},~\citep{Street_2020}).


\subsection{Community Forum}
\label{ssec:forum}
The Rubin Observatory Community Platform has a dedicated category for Early Science\footnote{See \url{https://community.lsst.org/t/about-the-early-science-category/5775}}, where community members are encouraged to open discussions on the topic of early science. 
Community feedback on the Early Science data products is welcomed and will help the Rubin to improve its data products and services. 
