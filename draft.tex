
Difference Image Analysis (DIA) subtracts new science mages  from transient-free coadded template images  to create a difference image. 
Sources detected in the difference images are reported within 60s of image readout in the form of a world-public stream of {\it Alert} packets, distributed to Community Brokers for further analysis and follow-up by the Rubin science community. 
LSST is expected to produce about 10 million alerts per night in {\it steady state} operations. 
The templates needed by DIA are produced as part of the annual \drp. 

This is due to the fact that {\it baselined} templates needed for \diffim are produced in \drp and \drone is not until one year after the start of operations.
Key components of the Early Science Program include
%\begin{itemize}
%\item Commissioning data obtained with LSSTCam, including at least 2 months of data obtained in sustained observing and made available as a available as a {\it Data}
%\item Brokered alerts from templated sky regions
%\end{itemize}

The esp wil deliver both DR data products to enable static science and prompt data products  for time-domain science 
DPs and Alerts  
The following strategy, as recommended by the  Rubin Construction Data Management (DM) Science team (DM-SST) and  Project Science Team (PST) has been adopted by Rubin Operation for Alert generation in year 1: 
Commissioning Data Templates: Build templates, where possible, from all commissioning data before the start of year one, and use them to generate alerts during year one.
? Year One Data Templates: Build templates progressively from data obtained during year
one (e.g., on a monthly timescale), and use them to generate alerts during year one,
either instead of, or in addition to using commissioning data to build templates.





Prior to \drone, scientific investigations may be undertaken with data obtained during the \svs (\citep{}), and with data obtained during the first year of operations and processed in Alert Production and Solar System Processing. 
Alerts of transient, variable, and/or moving objects are the only data product that will be immediately available (within 60 seconds of image readout) and publicly shareable, i.e not subject to a proprietary period. 

Templates can be built from commissioning data obtained with the LSST camera and used to generate alerts in year 1 for the corresponding field and filter. 

data processed in Alert Production and Solar System Processing as well as \dpvs of the commissioning data.


Alert Production requires high-quality template images of the sky, and thus full volume and full fidelity alerts will not be available until after \drone.
However, \ro is planning a process of incremental template generation to maximize the time-domain and solar system science achievable in the first year of operations.
The worldwide community is actively preparing to process the LSST alert stream and use it to generate groundbreaking scientific results. 
This note describes the \ro plan for ensuring \es. 
It is a living document that will evolve over the course of the remainder of the construction project and up until \drone. 

Science from the \rolsst will be enabled by two categories of data products: Prompt data products and Data Release data products (\citep{LPM-231}, \citep{LSE-163}). 
Data Release data products result from a coherent processing of the entire science dataset to date, and will be released as a series of 11 approximately annual data releases over the 10-year period of the survey (\citep{RDO-011}).
Prompt data products are generated continuously every observing night by the real-time Prompt Processing pipelines using Difference Image Analysis (DIA). 
Five-sigma detections will be reported within 60s of image readout in the form of a world-public stream of {\it Alerts}, distributed to Community Brokers for further analysis and follow-up by the Rubin science community.  
Alerts of transient, variable, and/or moving objects are the only data product that will be immediately available and publicly shareable, i.e not subject to a proprietary period.
The worldwide community is actively preparing to process the LSST alert stream and use it to generate groundbreaking scientific results. 
 
 
 
  DR1 will contain data products generated from approximately the first 6
months of the survey, while DR2 will contain data products from the whole first year.

Difference Image Analysis (DIA) requires transient-free coadded template images of the sky.
Templates are generated as part of the annual Data Release Processing (DRP), and thus in the current survey baseline, full volume, full fidelity Alert production can only begin in year 2 of operations. 

Some expectations of early science have been built around the validation phase of commissioning,  which is designed to include at least two months of sustained observing. 
Templates can be built from commissioning data obtained with the LSST camera and used to generate alerts in year 1 for the corresponding field and filter. 



The early science program will be steadily evolved throughout pre-operations as the SIT-COM activities take place, adapting to the state of the system and with an eye on verification and validation data sets being produced in commissioning.



A key component of the ESP is the design a template acquisition strategy during both commissioning and the early phases of the survey, to increase the 
is committed to	

\ro is committed to ensuring a high-impact Early Science Program



Science during the first year of Rubin Observatory / LSST operations is a high priority. 
Prior to \drone, scientific investigations may be undertaken with data processed in Alert Production and Solar System Processing as well as \dpvs of the commissioning data.
Alerts of transient, variable, and/or moving objects are the only data product that will be immediately available (within 60 seconds of image readout) and publicly shareable, i.e not subject to a proprietary period. 
Alert Production requires high-quality template images of the sky, and thus full volume and full fidelity alerts will not be available until after \drone.
However, \ro is planning a process of incremental template generation to maximize the time-domain and solar system science achievable in the first year of operations.
The worldwide community is actively preparing to process the LSST alert stream and use it to generate groundbreaking scientific results. 
This note describes the \ro plan for ensuring \es. 
It is a living document that will evolve over the course of the remainder of the construction project and up until \drone. 

-----
Prompt data products
Additionally, to enable early alert science, Rubin alert production will include the capability to build ``incremental'' templates from on-sky imaging as it becomes available during commissioning as well as the first year of operations. 
Such templates will be built periodically as images accumulate to allow for partial alert generation over an incomplete sky footprint.
How extensive these templates are at the start of full survey operations will be influenced on the overall success of commissioning including Science Validation surveys and other significant datasets. These same datasets will provide static sky data release-like products in the Data Previews. 

