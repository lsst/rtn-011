\section{Rubin Observatory Commissioning} 
\label{sec:commissioning}

\subsection{Schedule}
Outline the current schedule 



\subsection{Science-driven prioritization of sky templates}
By the end of the commissioning period, coadd templates for use in difference imaging will only be available for $\approx$ 10\% of the sky. 
This leaves open the question of how to  prioritize  sky areas and bandpasses to optimize the science harvest prior to \drone (\S \ref{sec:science}).

\subsection{Template generation during commissioning}


\subsection{Template verification in commissioning}

The LSST SRD places well-defined criteria on the quality of the difference image and the amount of noise that a template can contribute to a difference image.  These criteria result in a minimum of three images being needed to construct a template for use in year one.  The commissioning period provides an excellent opportunity to investigate how many visits in a given band are sufficient to construct a usable template.  Given the desire to maximize the science harvest prior to the \drone,  relaxing these criteria might be preferable. 

\subsection{Alert generation during commissioning}

Due to the need to verify the instrument characteristics, template quality, and image differencing and Real/Bogus performance, real-time alerts will not be immediately available during the commissioning period. 
Where the accumulated ComCam data is sufficient for alert generation, we expect to provide alerts at high latency (weeks--months). 
The goal for these ``canned'' alerts is to enable alert brokers and science users to understand their characteristics and to help to validate their quality rather than to enable rapid followup and \es per se.
During LSSTCam commissioning we intend to generate incremental templates over the maximal sky area supported by the available observations.
These templates will be used for Alert Production, with an aim of approaching near-real-time alert distribution to community brokers by the time of the Science Validation Surveys at the end of LSSTCam commisioning. 
