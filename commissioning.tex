\section{Rubin Observatory Commissioning and Early System Optimization}
\label{sec:commissioning}

Rubin Observatory commissioning was completed on \cstrcompdate, and the Observatory is now in an \emph{Early Optimization} phase. Figure~\ref{fig:commissioning-es-schedule} presents the detailed schedule of completed commissioning and ongoing Early Science activities relative to the start of Operations.
\begin{figure}[htb]
\centering
\includegraphics[width=0.98\linewidth]{rubinobs_on-sky_commissioning_and_early_science}
\caption{Detailed schedule of commissioning  and early science activities relative to Rubin First Light, as of \currentdate.}
\label{fig:commissioning-es-schedule}
\vspace{0.1cm}
\end{figure}

\subsection{Summary of Commissioning Activities}
\label{ssec:commissioning-schedule}

The completion of Rubin Observatory commissioning was marked by a series of key milestones, which are included here for completeness.

\textbf{LSSTComCam First Photon}: The first image of the night sky produced by photons passing through the Rubin optical system and detected by the Commissioning Camera (LSSTComCam). 
This milestone was achieved on \ccfpdate. 

\textbf{LSSTCam First Photon}: The first image of the night sky produced by photons passing through the Rubin optical system and detected by the LSST Science Camera (LSSTCam).
This milestone was achieved on \lcfpdate. 

\textbf{Rubin First Light}: The point at which Rubin routinely began to acquire science-grade imaging across the LSSTCam full focal plane.
This milestone was achieved on \rfldate. 

\textbf{Rubin Construction Commissioning Complete}: The point at which Rubin Observatory was deemed sufficiently complete and functional to be handed over from Construction to Operations for early system optimization and preparation for the start of the LSST survey. 
This milestone was achieved on \cstrcompdate.

Commissioning data collection was carried out in a sequence of planned phases, shown in Figure~\ref{fig:commissioning}, beginning with on-sky engineering using LSSTComCam and concluding with LSSTCam. 
The System Optimization and Science Validation (SV) phases, both conducted with LSSTCam, focused on refining system performance and verifying scientific readiness through targeted observations. 
These observations adopt many of the design elements of the standard LSST cadence, with modifications to increase the likelihood of delivering a stand-alone high-impact dataset to enhance opportunities for Early Science. 
Field selection was guided by commissioning priorities and community input, balancing technical constraints with scientific opportunity. 
Collectively, these efforts validated the end-to-end performance of the as-built system, confirmed its ability to deliver seeing-limited image quality, and demonstrated substantial construction completeness
\begin{figure}[hbt]
\centering
\includegraphics[width=0.95\linewidth]{commissioning-plan}
\caption{High-level plan that guided the collection of commissioning data.}
\label{fig:commissioning}
\end{figure}


\subsection{LSSTComCam Commissioning}
\label{ssec:commissioning-comcam}

LSSTComCam is Rubin's engineering camera that is used for testing and validating the observatory's systems and processes prior to installation of the LSST Camera.
The LSSTComCam focal plane has single raft with a 3×3 mosaic of 4K×4K ITL science sensors, giving a total of 144Mpix, 
LSSTComCam has the same plate scale as LSSTCam (0.2 arcsec / pixel), with a field of view of 40 × 40 arcmin.
The LSSTComCam filter exchanger holds only three physical filters at a time.

The Rubin on-sky commissioning campaign using  LSSTComCam began on 24 October 2024 and ended on 11 December 2024, lasting a total of 7 weeks, and included observations to support both engineering and science pipelines commissioning.
This highly successful campaign included a first series of on-sky engineering tests demonstrating the end-to-end functionality of the Simonyi Survey Telescope’s hardware and software systems  LSSTComCam.
The median delivered image quality for commanded in-focus images collected during the campaign, quantified in terms of the PSF FWHM, was $\approx1.1$ arcseconds. 
The best images have delivered PSF FWHM of $\approx0.7$ arcseconds.
A full report on the  LSSTComCam on-sky commissioning campaign is available at \citeds{SITCOMTN-149}.


\subsection{LSSTCam Commissioning}
\label{ssec:commissioning-lsstcam}

LSSTCam was installed on the Simonyi Survey Telescope on \currentdate. 
Following initial engineering and integration work and achievement of the LSSTCam First Photon milestone, the  on-sky commissioning campaign using the LSSTCam began on 15 April 2025 and ended on 21 September 2025. 

As summary of the SV data obtained using the Feature based Scheduler can be found at \url{https://survey-strategy.lsst.io/progress/sv_status/sv_20250930.html}. 
Figure \ref{fig:lsstcam_sv_nvisits} shows a map of all the science visits acquired during LSSTCam commissioning.
\begin{figure}[hbt]
\centering
\includegraphics[width=0.95\linewidth]{lsstcam_sv_nvisits}
\caption{All science visits acquired during LSSTCam commissioning.}
\label{fig:lsstcam_sv_nvisits}
\end{figure}



A full report on the  LSSTCam on-sky commissioning campaign is in preparation at \citeds{SITCOMTN-170}.
