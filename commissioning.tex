\section{Rubin Observatory Commissioning and Early System Optimization}
\label{sec:commissioning}

Rubin Observatory commissioning was completed on \cstrcompdate, and the Observatory is now in an \emph{Early Optimization} phase. Figure~\ref{fig:commissioning-es-schedule} presents the detailed schedule of completed commissioning and ongoing Early Science activities relative to the start of Operations.
\begin{figure}[htb]
\centering
\includegraphics[width=0.98\linewidth]{rubinobs_on-sky_commissioning_and_early_science}
\caption{Detailed schedule of completed commissioning and ongoing Early Science activities relative to the start of Operations.}
\label{fig:commissioning-es-schedule}
\vspace{0.1cm}
\end{figure}

\subsection{Summary of Commissioning Activities}
\label{ssec:commissioning-schedule}

The completion of Rubin Observatory commissioning was marked by a series of key milestones, which are included here for completeness.

\textbf{LSSTComCam First Photon}: The first image of the night sky produced by photons passing through the Rubin optical system and detected by the Commissioning Camera (LSSTComCam). 
This milestone was achieved on \ccfpdate. 

\textbf{LSSTCam First Photon}: The first image of the night sky produced by photons passing through the Rubin optical system and detected by the LSST Science Camera (LSSTCam).
This milestone was achieved on \lcfpdate. 

\textbf{Rubin First Light}: The point at which Rubin routinely began to acquire science-grade imaging across the LSSTCam full focal plane.
This milestone was achieved on \rfldate. 

\textbf{Rubin Construction Complete}: The point at which Rubin Observatory was deemed sufficiently complete and functional to be handed over from Construction to Operations for early system optimization and preparation for the start of the LSST survey. 
This milestone was achieved on \cstrcompdate.

Commissioning data collection was carried out in a sequence of planned phases, shown in Figure~\ref{fig:commissioning}, beginning with on-sky engineering using LSSTComCam and concluding with LSSTCam. 
The System Optimization and Science Validation (SV) phases, both conducted with LSSTCam, focused on refining system performance and verifying scientific readiness through targeted observations. 
These observations adopted many of the design elements of the standard LSST cadence, with modifications to increase the likelihood of delivering a stand-alone high-impact dataset to enhance opportunities for Early Science. 
Field selection was guided by commissioning priorities and community input, balancing technical constraints with scientific opportunity. 
Collectively, these efforts validated the end-to-end performance of the as-built system, confirmed its ability to deliver seeing-limited image quality, and demonstrated substantial construction completeness
\begin{figure}[hbt]
\centering
\includegraphics[width=0.95\linewidth]{commissioning-plan}
\caption{High-level plan that guided the collection of commissioning data.}
\label{fig:commissioning}
\end{figure}


\subsection{LSSTComCam Commissioning}
\label{ssec:commissioning-comcam}

LSSTComCam is Rubin's engineering camera that was used for testing and validating the observatory's systems and processes prior to the installation of the LSST Science Camera.
The LSSTComCam focal plane has single raft with a 3×3 mosaic of 4K×4K ITL science sensors, giving a total of 144Mpix, 
LSSTComCam has the same plate scale as LSSTCam (0.2 arcsec / pixel), with a field of view of 40 × 40 arcmin.
The LSSTComCam filter exchanger holds only three physical filters at a time.

The Rubin on-sky commissioning campaign using  LSSTComCam began on 24 October 2024 and ended on 11 December 2024, lasting a total of 7 weeks, and included observations to support both engineering and science pipelines commissioning.
This highly successful campaign included a first series of on-sky engineering tests demonstrating the end-to-end functionality of the Simonyi Survey Telescope’s hardware and software systems.
The median delivered image quality for commanded in-focus images collected during the campaign, quantified in terms of the PSF FWHM, was $\approx1.1$ arcseconds. 
The best images have delivered PSF FWHM of $\approx0.7$ arcseconds.
A full report on the  LSSTComCam on-sky commissioning campaign is available at \citeds{SITCOMTN-149}.


\subsection{LSSTCam Commissioning}
\label{ssec:commissioning-lsstcam}

LSSTCam was installed on the Simonyi Survey Telescope on \currentdate. 
Following initial engineering and integration work and the achievement of the LSSTCam First Photon milestone, on-sky commissioning campaign using the LSSTCam began on 15 April 2025 and ran through  21 September 2025. 

The acquisition of science images\footnote{As opposed to images taken for engineering or Active Optics System (AOS) testing.} with LSSTCam began on 17 April 2025.
Early observations consisted of  small-field survey visits dithered over small areas around a central boresight. 
These visits were typically acquired in sequences of at least 10 visits per bandpass, cycling through the available filters, in a manner similar to the small-field survey science visits acquired during LSSTComCam commissioning and  included in DP1.

The Science Validation (SV) surveys began on 2025-06-20, acquiring visits in a manner consistent with the planned LSST operations cadence. 
While based on the baseline LSST survey design, the SV surveys incorporated several modifications to maximize the likelihood of producing a stand-alone, high-impact dataset that would enhance opportunities for Early Science. 
They covered a more limited sky area to achieve greater depth within the SV timeframe, resulting in a temporal sampling distribution that differs from the LSST baseline.
Figure \ref{fig:lsstcam_sv_nvisits} shows a sky map of all the science visits acquired during LSSTCam commissioning.
\begin{figure}[hbt]
\centering
\includegraphics[width=0.95\linewidth]{lsstcam_sv_nvisits}
\caption{All science visits acquired during LSSTCam commissioning.}
\label{fig:lsstcam_sv_nvisits}
\end{figure}

The SV surveys were executed using the Feature based Scheduler (FBS) and comprised two primary components, interleaved within a single FBS configuration.
\begin{itemize}
\item \textbf{Wide Survey:}  Designed to test template generation and Prompt Processing with difference image analysis at data rates representative of the first year of the LSST. This provided a sustained full-scale test of the Data Facility;
\item \textbf{Deep Survey:} Designed to test the production of deep coadds with integrated exposures  equivalent to or exceeding those of the LSST 10-year survey.  These observations achieved a rapid temporal sampling in the selected deep fields and validated the observing strategy for the LSST Deep Drilling Fields (DDFs).
\end{itemize}
The Wide Survey traced the ecliptic plane from dense regions of the Galactic Bulge through low-dust regions within the planned LSST Wide-Fast-Deep (WFD) footprint.
Four of the planned LSST Deep Drilling Fields (DDFs) were observed as part of the Deep Survey. 
A secondary area within the low-dust WFD footprint was included to provide alternate targets when the primary or DDF fields were  unavailable, supporting early template generation in higher-declination areas.

A total of 21,647 science visits were acquired during the SV surveys. 
This total excludes known bad visits (as of 2025-09-30) but includes exposures spanning a broad range of data quality, reflecting variations in cloud extinction, delivered image quality, and engineering issues. 
Of these, 7,194 were obtained in small-field survey mode, 908 were targeted at the Deep Drilling Fields (DDFs), and 13,240 covered the primary wide SV area. 
Additionally, 194 visits acquired for Target of Opportunity (ToO) testing purposes
Most of the visits were acquired between 2025-04-17 and 2025-07-24, with interruptions due to weather and engineering activities required to improve delivered image quality.
Figure \ref{fig:lsstcam_svvisits_timing} shows the temporal distribution LSSTCam SV survey visit acquisition.
\begin{figure}[hbt]
\centering
\includegraphics[width=0.95\linewidth]{lsstcam_scivisits_timing}
\caption{Timeline of LSSTCam commissioning science visit acquisition.}
\label{fig:lsstcam_svvisits_timing}
\end{figure}

Image quality proved challenging throughout the SV period
This is evidenced in part as a point-spread function (PSF) larger than expected given estimates of the atmospheric contribution, and partly as spatial variations in the PSF across the field of view.
While the median-per-visit PSF width at FWHM reported for the SV images is not substantially larger than  typical seasonal values,  the shape and variability of the PSF do not meet the performance requirements expected for full survey operations.

As full summary of the SV data obtained during LSSTCam commissioning can be found at \url{https://survey-strategy.lsst.io/progress/sv_status/sv_20250930.html}. 
A full report on the  LSSTCam on-sky commissioning campaign is in preparation at \citeds{SITCOMTN-170}.


% Small field surveys
\subsubsection{Small Field Surveys}
\label{sec:small_field}
Several small fields were observed prior to the start of the SV surveys, with some overlapping the early portion of the SV survey period. 
A subset of these observations formed part of the Rubin First Look campaign.
A summary of the data collected during the small-field surveys is provided in Table \ref{tab:sv_small_field_summary}. 
The median Image Quality (IQ) is expressed in terms of the width of the PSF at FWHM.
\begin{table*}[htbp]
\centering
\renewcommand{\arraystretch}{1.2}
\begin{tabular}{lccr}
\toprule
\textbf{Field} &
\textbf{N} &
\textbf{Median IQ} & 
\textbf{Timespan} \\
 & Visits & \textbf{(arcsec)} & \textbf{(days)} \\
\midrule
Carina (NGC 3372)             & 124  & ---  & 4 \\
Rubin\_SV\_280\_\-48 & 148  & 1.52 & 1 \\
ELAIS\_S1           & 166  & 1.38 & 17 \\
Rubin\_SV\_320\_\-15 & 273  & 1.22 & 6 \\
New\_Horizons       & 360  & 1.05 & 62 \\
Rubin\_SV\_216\_\-17 & 386  & 1.29 & 8 \\
Rubin\_SV\_212\_\-7  & 498  & 1.21 & 7 \\
Prawn               & 632  & 1.37 & 79 \\
COSMOS              & 664  & 1.22 & 15 \\
Triffid (M20) \&  Lagoon (M8)       & 668  & 1.14 & 10 \\
M49                 & 1173 & 1.29 & 13 \\
Rubin\_SV\_225\_\-40 & 2052 & 1.36 & 97 \\
\bottomrule
\end{tabular}
\caption{Summary of selected small-field observations showing the number of visits, median delivered image quality (PSF FWHM), and total timespan of observations.}
\label{tab:sv_small_field_summary}
\end{table*}


% Wide area summary 
\subsubsection{SV Wide Area}
\label{sec:sv_wide}
The initial SV wide area was approximately 3000 deg$^2$; this was reduced to 750 deg$^2$ approximately halfway through the SV survey period. 
This was further reduced to approximately 300 deg$^2$ during the last week or so of the SV period  to concentrate observing cadence and  better enable image differencing tests. 
Table \ref{tab:sv_wide_configs} provides the median numbers of visits and estimated m5 coadded depths  within each of the subsets of the Wide SV area.
The median image quality expressed in terms of the width of the PSF at FWHM is also provided per band across all subset areas in the SV Wide Area survey.
\begin{table*}[htbp]
\centering
\renewcommand{\arraystretch}{1.2}
\begin{tabular}{lccccccc}
\toprule
\textbf{Field} &
\multicolumn{6}{c}{\textbf{Band}} &
\textbf{Total} \\
\cmidrule(lr){2-7}
 & $u$ & $g$ & $r$ & $i$ & $z$ & $y$ & \textbf{N visits} \\
\midrule
3k deg$^2$  & 2 & 4 & 7 & 8 & 7 & 5 & 38 \\
             & 24.2 & 25.0 & 24.9 & 24.5 & 23.8 & 22.6 &  \\
\addlinespace
750 deg$^2$ & 2 & 4 & 12 & 16 & 11 & 9 & 56 \\
             & 24.4 & 25.2 & 25.2 & 24.8 & 24.0 & 23.0 &  \\
\addlinespace
300 deg$^2$ & 2 & 5 & 11 & 18 & 15 & 10 & 64 \\
             & 24.5 & 25.2 & 25.0 & 24.9 & 24.1 & 23.0 &  \\
\midrule
\makecell[l]{\textbf{Median FWHM}\\\textbf{(arcsec)}} 
 & 1.18 & 1.26 & 1.26 & 1.24 & 1.17 & 1.25 &  \\
\bottomrule
\end{tabular}
\caption{Median number of visits per pointing and estimated $m_5$ coadded depths per band within each subset of the wide-area SV survey. The final row gives the median FWHM of the point spread function in arcseconds.}
\label{tab:sv_wide_configs}
\end{table*}

Figure \ref{fig:wide-image-quality} shows  distributions of the image quality and depth for the SV Wide Area Survey data 
\begin{figure}[htbp]
    \centering
    \begin{subfigure}[b]{0.465\textwidth}
        \centering
        \includegraphics[width=\textwidth]{sv_wide_fwhm}
        \caption{Delivered image quality.}
        \label{fig:subfig1}
    \end{subfigure}
    \hfill
    \begin{subfigure}[b]{0.48\textwidth}
        \centering
        \includegraphics[width=\textwidth]{sv_wide_m5}
        \caption{Estimated individual image depth.}
        \label{fig:subfig2}
    \end{subfigure}
    \caption{Summary of the delivered image quality per visit and  estimated individual image depth per visit for visits in the  SV Wide Area survey. } 
    \label{fig:wide-image-quality}
\end{figure}
The individual image depth across the SV wide-area survey  is slightly shallower than predicted by  baseline survey simulations, which model the  full LSST  footprint over the 10 years of operations. 
This difference can be attributed to poorer delivered image quality and slightly higher mean cloud extinction during the SV surveys. 
Under favorable conditions, the measured median zeropoints for individual images were consistent with predicted values, indicating that the overall system sensitivity was generally as expected.

% DDF area summary 
\subsubsection{Deep Drilling Fields}
The four DDFs observed during the survey had variable completeness.
Table \ref{tab:sv_ddf_summary}  lists the number of visits obtained during LSSTCam commissioning, including observations acquired during the  small-field  and SV surveys, within each Deep Drilling Field (DDF). 
For each field, the first row gives the number of visits per band, while the second row reports the corresponding estimated coadded $5\sigma$ depths ($m_5$). 
The total number of visits, the timespan over which observations were collected, and the median image quality, expressed in terms of the PSF at FWHM are also provided.
\begin{table*}[htb]
\centering
\renewcommand{\arraystretch}{1.2}
\begin{tabular}{lccccccccc}
\toprule
\textbf{Field} &
\textbf{Timespan} &
\textbf{Median FWHM} &
\multicolumn{6}{c}{\textbf{Band}} &
\textbf{Total} \\
\cmidrule(lr){4-9}
 & (days) & (arcsec) & $u$ & $g$ & $r$ & $i$ & $z$ & $y$ & \textbf{N visits} \\
\midrule
XMM\_LSS  & 10 & 1.47 & 30 & — & — & — & — & — & 30 \\
           &    &      & 25.3 & — & — & — & — & — &  \\
\addlinespace
EDFS\_A   & 55 & 1.32 & — & 20 & 21 & 28 & 18 & — & 87 \\
           &    &      & — & 25.3 & 25.2 & 25.0 & 24.1 & — &  \\
\addlinespace
EDFS\_B   & 55 & 1.37 & — & 23 & 21 & 29 & 18 & — & 91 \\
           &    &      & — & 25.4 & 25.1 & 25.0 & 24.1 & — &  \\
\addlinespace
ECDFS     & 57 & 1.51 & — & 36 & 39 & 41 & 43 & — & 159 \\
                 &      &         & — & 25.8 & 25.5 & 25.3 & 24.7 & — &  \\
\addlinespace
ELAISS1   & 84 & 1.40 & 39    & 101 & 103 & 168 & 112 & 16 & 539 \\
                 &       &         & 25.6 & 26.6&  26.1 &  26.2 &  25.1 & 22.7 & \\        
\addlinespace
COSMOS    & 15 & 1.22 & 100 & 82 & 166 & 139 & 111 & 66 & 664 \\
                    &      &        & 26.0 & 26.8 & 26.3 & 26.1 & 25.3 & 23.9 &  \\
\bottomrule
	
\end{tabular}
\caption{Summary of Deep Drilling Field observations from LSSTCam on-sky commissioning data, including SV and small-field campaigns. For each field, the first row lists the number of visits per band and total, and the second row lists the corresponding coadded $5\sigma$ depths ($m_5$). The observation timespan and median image FWHM are also indicated.}
\label{tab:sv_ddf_summary}
\end{table*}

The SV surveys occurred relatively early in the observing season for all Deep Drilling Fields, and COSMOS was entirely unavailable for inclusion. 
The COSMOS data in Table~\ref{tab:sv_ddf_summary} therefore originate exclusively from the Small Fields Surveys (\S~\ref{sec:small_field}).
The remaining DDFs became observable toward the end of the night beginning in July, with ELAISS1 rising first. 
Consequently, ELAISS1 received the largest number of visits, while the other DDFs were observed less extensively due to limited SV on-sky time.
ELAISS1 was the only field observed in all six bands, $ugrizy$, during the SV survey; COSMOS also received observations in all bands as part of the Small Fields Surveys.
Most of the others were observed only in $griz$, as these were the only available filters after 2025-08-10 due to hardware issues with the filter exchanger. 
The XMM-LSS field received visits only in $u$, owing to a combination of factors including the compressed observing cadence, competition for  observing time between the DDFs, azimuth constraints at the end of the night, patchiness of SV observing time, and periodic unavailability caused by moon avoidance.

The dither pattern used for the DDFs evolved over the course of the SV Surveys. 
Initially, only small dithers were used, and no intra-night dithering was used. 
This configuration led to challenges with scattered light and the absence of intra-night dithers limited the ability to construct high-quality template images. 
Later in the campaign, both translational and rotational dithers were introduced within a night, while inter-night dithering was maintained. 
The optimal dither pattern for LSST DDF observations is the subject of ongoing investigations,  but will likely include both intra-night and inter-night dithering components.