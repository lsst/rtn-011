\section{Rubin Observatory Commissioning} 
\label{sec:commissioning}

The various different science drivers outlined in \ref{sec:science} naturally lead to different priorities for template generations, e.g. solar system science prefers templates to be generated in the NES and Milky Way science would to would prefer templates for the galactic plane to optimise alert production in these areas in early operations. Other science will prefer templates in a number of filters to enable .. rather that larger area. 

\subsection{Science-driven prioritization of sky templates}
By the end of the commissioning period, coadd templates for use in difference imaging will only be available for $\approx$ 10\% of the sky. 
This leaves open the question of how to  prioritize  sky areas and bandpasses to optimize the science harvest prior to \drone.

\subsection{Template verification in commissioning}

The LSST SRD places well-defined criteria on the quality of the difference image and the amount of noise that a template can contribute to a difference image.  These criteria result in a minimum of three images being needed to construct a template for use in year one.  Science collaborations The commissioning period provides an excellent opportunity to investigate how many visits in a given bad are sufficient to construct a template that is good enough.  Given the desire to maximize the science harvest prior to the \drone,  relaxing these criteria might be preferable. 
