\section{Elements of the Early Science Program}

The Early Science Program consists of the following elements:
\begin{itemize}
	\item A series of three \emph{Data Previews} based on either simulated LSST-like data or data taken during the Rubin Observatory commissioning period with the LSST Science Camera (LSSTCam).
	\item A world-public stream of alerts from transient, variable, and moving sources that will be scaled up during commissioning and the first year of the survey.
	\item Template generation, both prior to the start of regular survey operations based on data collected during the commissioning period with LSSTCam, and incrementally during the first year regular survey operations  to maximize the number of templates available for Alert Production in year 1.
	\item LSST Data Release 1 (DR1), which will be based on the Data Release Processing (DRP) of the first six months of LSST data.
\end{itemize}

\subsection{Data Previews}

The twin goals of the Data Previews are 1) to familiarize the community with LSST analysis in advance of the first survey data release, and b) to help drive development of Rubin operations capabilities prior to survey start.
The images for both Data Preview 1 (DP1) and Data Preview (DP2) will be taken with LSSTCam, since (as of November 2022) the Commissioning Camera (ComCam) is no longer scheduled to have any on-sky observation time.

Three pre-survey data previews are planned:
\begin{itemize}
\item  Data Preview 0 (DP0): Based on simulated LSST-like data
\item  Data Preview 1: Based on a few nights of early science-grade commissioning data taken with LSSTCam
\item Data Preview 2: Based on a full reprocessing of all science-grade LSSTCam data.

\end{itemize}

The first two phases of Data Preview 0 (DP0) were released to a group of early adopters from the community in the summers of 2021 and 2022, and are based on the DESC DC2 simulated dataset, \citep{2021ApJS..253...31L}. A third phase, focusing on simulated solar system science catalogs, is planned for release in summer 2023.


Subsequent Data Previews, DP1 and DP2 will be based on the data acquired during Rubin commissioning with LSSTCam.

\subsection{Alert Production}

A detailed description of the Alert Production is given in \S~\ref{sec:pp}

\subsection {Template Generation}

A key component of the Early Science Program is the generation of templates from both the commissioning data and data collected during the first year of the survey.
In full survey operations, template images for difference image analysis and alert generation are constructed as part of the annual DRP.
In order to support alert generation in year 1, Rubin will generate templates from all science-grade data taken during commissioning to provide an initial template library at the start of the 10-year survey, and the incrementally generate templates during year 1 using the best images available and covering as much sky in as many filters as possible.
Details of the current strategy for alert generation  with incremental templates are given in  \S~\ref{sec:pp}.

\subsection{Data Release 1}

LSST Data Release 1 will be based on the first six months of data taken as part of the 10-year survey.
Data Release Processing is estimated to
