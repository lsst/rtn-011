\section{Alert Production in Year One}
The RUbin Construction Data Management (DM) Science team (DM-SST), carried out a study, of several options for Alert Production in Year 1, reported in DMTN-107: Options for Alert Production in LSST Operations Year 1.  
Representatives of the Rubin Project Science Team (PST), DM-SST and Operations reviewed the proposed DM-SST options at a meeting in October, 2019 and converged on a proposed strategy for Alerts in year 1:

\begin{itemize}
\item Commissioning – Data 
Templates: Build templates, where possible, from all commissioning data before the start of year one, and use them to generate alerts during year one. 
\item Year One – Data Templates: Build templates progressively from data obtained during year one 
(e.g., on a monthly timescale), and use them to generate alerts during year one, 
either instead of, or in addition to using commissioning data to build templates.
\end{itemize}

To handle alert generation outside the template building process attached to theannual DRP, the Construction project initiated a change request to include incremental templates in the DM system workflow. This change has been accepted and is now part of the baselined DM project in constructiom. A summary of the changes is the following:

\begin{itemize} 
\item LCR-2273: Construct Image Differencing Templates Outside DRP, new requirment 1.4.6 Template Coadds ID: DMS-REQ-0280, The DMS shall periodically create Template Images in each of the u,g,r,i,z,y passbands. Templates may be constructed as part of executing the Data Release Production payload, or by a separate execution of the Template Generation payload. Prior to their availability from Data Releases these coadds shall be created incrementally when sufficient data passing relevant quality criteria is available. 
\item To enable artifact rejection, templates will be built with at least three images in year one, and five in subsequent years.  (Rubin OSS-REQ-0158)
\item Once a template is produced for a sky position and filter it will not be replaced until the next Data Release to avoid repeated baseline changes.
\item Templates are not necessarily built from the first N images that are collected.
\end{itemize}

