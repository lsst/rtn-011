
%During routine LSST operations, prompt image data products will be made available 80 hours following camera readout.
%They include raw images, processed single visit images (PVIs), difference images, and template images.
%Access to unvetted PVIs and difference images in the first 6 months of the LSST is still to be decided.
\section{Prompt Data Products}
\label{sec:prompt-products}

Prompt Data Products are generated by the Prompt Processing pipelines on timescales ranging from 120 seconds to 80 hours after each observation. 
Their primary purpose is to enable the rapid discovery and characterization of transient, variable, and moving objects, supporting timely community follow-up of time-domain events.
They complement the annually produced Data Release Products by providing near real-time access to time-domain information during nightly operations.

These products include single-visit images, difference images, and catalogs of detections in difference images (\textit{DiaSources}), their associated astrophysical objects (\textit{DiaObjects}), and Solar System objects (\textit{SSObjects}).
Alerts for newly detected \textit{DiaSources} are issued within 120 seconds of observation using community-standard formats and distributed to Community Brokers to facilitate rapid follow-up.

The \DPDD{} summarizes the pipelines used to generate the Prompt Data Products.
Rubin plans to progressively scale up access to these products during the Early Science period.

%%% Alerts
%% Eric -- please update to provide the status of templates and early alerts as of Nov 2025
\subsection{Alert Production in Commissioning and Early Operations}
\label{sec:pp}

Both Alert Production and Solar System Processing depend on the existence of template images.
During steady-state operations, these templates will be constructed during the annual Data Releases and will be built from the best available subset of images.
To enable alert production to proceed during commissioning and early operations, it is necessary build templates incrementally as data become available, as recommended by the study described in \citeds{DMTN-107}.
Because we have a smaller set of input images to choose from and uncertain knowledge about future observations, incremental template generation necessarily must balance the trade-off of earlier template availability against template quality and spatial completeness.
Validation will be required to determine when to build incremental templates to maximize the net throughput of Early Science.
Nevertheless, our goal is to enable Alert Generation to begin over at least a subset of the survey area as soon as the data are scientifically useful.

Scientifically it is important that once a template is constructed for a given region of sky, it is used exclusively until it can be updated in the next Data Release.
Repeated changes to the template make it extremely difficult to construct usable lightcurves for objects from individual difference image sources: transient objects such as supernovae will be contaminated by changing flux levels from the evolving template, and variable objects such as variable stars and AGN will require repeated corrections for different template flux levels as well.

During commissioning templates will be generated incrementally over the maximal sky area supported by the available observations.
By the end of the commissioning period, coadd templates for use in difference imaging will only be available for $\approx$ 10\% of the sky.
Generating templates over a wide area is not an explicit goal of commissioning;  however, where possible, if commissioning observations are agnostic to pointing and filter, we would endeavour to choose a pointing and filter that maximizes building templates to enable early science.

Rubin aims to scale up alert production during commissioning with the aim of beginning early Alert Production over a progressively increasing fraction of the sky as soon as is feasible following Rubin First Light  (\S~\ref{sec:timeline}).
\citeds{RTN-061} describes the criteria for sending the first Rubin alerts.
Once begun, Alert Production will then proceed continuously into the full LSST survey.
Alerts generated during the early science period may be produced with higher latency, and access to images and the PPDB (\S~\ref{sec:ppdb}) may not be available during this phase.
During  commissioning and early Operations periods,  alert packets for moving objects might not include the associated historical source records , and parameters such as phase curve slope (G) would be empty until sufficient detections exist to derive them.

The first Rubin alerts are currently expected to begin between December 2025 and February 2026. 

%%% Prompt images
\subsection{Prompt Images and Catalogs}
\label{sec:prompt-images}

Starting in early operations and continuing through Data Release 1 (DR1), Rubin plans to begin nightly releases of single-epoch Processed Visit Images (PVIs) together with the associated Source catalogs of detections. 
These Source catalogs will be provided through the Butler on a per-image basis.
The PVIs will be subject to the standard 80-hour embargo period, while the catalogs are not but, for convenience and consistency, will be released together with the images after the 80-hour period following shutter close. 
During the first six months of LSST operations, prompt PVIs and difference images may be released with somewhat higher latency as Rubin continues to assess data quality and scale up data services.

Prompt image publication is under active development.  The current expected start date for nightly PVI and catalog releases is not before April 2026. 

\subsection{Prompt Products Database}
\label{sec:ppdb}

The Prompt Products Database (PPDB) serves as Rubin’s real-time catalog of transient, variable, and moving-object detections, supporting community use of alerts.
It stores the Catalog Prompt Data Products and is designed to provide efficient access for scientific analysis and querying.

Due to required technical re-engineering, the PPDB is currently expected to become available around July 2026.