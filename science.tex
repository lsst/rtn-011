\section{Science Drivers} \label{sec:science}

The various different science drivers outlined in \ref{sec:science} naturally lead to different priorities for template generations, e.g. solar system science prefers templates to be generated in the NES and Milky Way science would to would prefer templates for the galactic plane to optimise alert production in these areas in early operations. Other science will prefer templates in a number of filters to enable .. rather that larger area. 

\subsection{Time Domain}

The \tvssc reviewed the opportunities for \es for non time-critical and  time-critical science cases in \cite{Hambleton_2020} and \cite{Street_2020} respectively. 

\subsection{Solar System}

The \sssc reviewed opportunites for \es in \citeds{2020arXiv201005926L}. 
LSST is predicted to discover $\approx$ 6 million solar system planetesimals, providing in total over a billion photometric and astrometric measurements in 6 broad-band filters. 

\subsection{Static Science}
The baseline static science data sets will flow from \sv surveys carried out during commissioning. 

\subsection{Target of Opportunity}
Rubin Observatory will be prepared to take advantage of Targets of Opportunties (TOO) in the first year of operations (and hopefully SIT-COM). 
\citedsp{RTN-008} describes potential data processing scenarios for TOO observations in the early operations era.