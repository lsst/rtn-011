\section{Science Considerations for Optimizing Early Science} \label{sec:science}

It will not be possible to survey the whole sky in all filters and generate templates by the end of the commissioning period.
A strategy for template generation in the early phases of the survey, which will require balancing a tradeoff between various factors such as smaller area with multiple filters vs a single filter over a large area, must be devised.
Different science drivers naturally lead to different prioritization strategies, e.g. Milky Way science would prefer templates that cover the Galactic Plane, time domain science would prefer templates in multiple bands rather than a single band for a larger area.
Supernova, transient and variable science strongly advocate for templates for all bands in the Deep Drilling Fields to be prioritized.
Rubin Operations will work closely with the science community to develop a science-driven approach to template generation in the early phases of the survey that will benefit the maximum number of science cases.

\subsection{Time Domain}

The \tvssc reviewed the opportunities for Early Science  for non time-critical and time-critical science cases in \citep{Hambleton_2020} and \citep{Street_2020} respectively.
In both cases, they recommend the prioritization of template acquisition in multiple bands as the preferred strategy rather than single-band  coverage over a larger area of sky.

\subsection{Solar System}

The \sssc reviewed opportunities for Early Science  in \citep{2020arXiv201005926L} for several high impact solar system science opportunities that would be enabled by accelerated template generation and alert production in year 1.
They find that template generation options that maximize the sky coverage in year 1 where LSST Solar System Processing can run daily are strongly preferred, even if the templates result in noisier image subtraction compared to later years.

\subsection{Static Science}

Datasets for static science will flow from the SV Surveys carried out during commissioning and released as Data Preview 2 (DP2).
The commissioning team are planning to acquire on-sky observations that would enable science validation studies for the four LSST science drivers.
Guidance is being sought from the community to enhance opportunities for science validation and early science  based on commissioning data.
Rubin Obs SIT-Com collected ``Commissioning Notes'' from the community in 2012 that are being considered as part of the
planning for the on-sky observing strategy during commissioning. \footnote {See https://community.lsst.org/t/community-input-to-the-on-sky-observing-strategy-during-commissioning/4406}


\subsection{Target of Opportunity}

Rubin Observatory will be prepared to take advantage of Targets of Opportunities (TOO) in the first year of operations (and hopefully SIT-Com).
\citedsp{RTN-008} describes potential data processing scenarios for TOO observations in the early operations era.
