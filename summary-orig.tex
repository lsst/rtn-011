\section{Summary}

Rubin Observatory is putting in place a dedicated {\it Early Science Program} to ensure high-impact science prior to the release of the first year of \lsst data.

This Early Science Program consists of X components, a series of \emph{Data Previews}

A series of  three \emph{Data Previews} , 2) 3) 4) 
A series of  three \emph{Data Previews} in which LSST-like data products are prepared, released to the LSST data rights community, and supported.
The goals of these Data Previews are to 1) enable the Operations teams to develop operational capability prior to the start of the LSST, and 2) support the members of the LSST science community as they develops their LSST analyses.
The first Data Preview in the series, DP0, makes use of simulated LSST-like data products and contains three phases.
DP0.1 was released in June 2021 and contains 300 \sqdeg of simulated, LSST-like images and catalogs generated by the Dark Energy Science Collaboration (DESC) for their Data Challenge 2, DC2 (\citep{2021ApJS..253...31L}, \url{https://dp0-1.lsst.io/}).
DP0.2, released 1 year later in June 2022, is based on the same DESC DC2 simulated dataset used for DP0.1, reprocessed by Rubin Observatory with a more recent version of the LSST Science Pipelines, and which additionally includes data products from Difference Image Analysis. 
DP0.3, currently planned for mid to late 2023, will serve simulated LSST-like catalogs specifically aimed at supporting Solar System science. 
Subsequent Data Previews, DP1 and DP2, will both be based on science-grade images taken with the LSST Science Camera (LSSTCam) during commissioning. 
DP1, anticipated for 2-3 months after System First Light, will make available the scientific data products associated with the System First Light images. 
DP2 is planned for 5-8 months after System First Light and will include a full processing of all science-grade images taken during commissioning. 
The data taken during the first 6 months of the LSST will be processed and released as Data Release 1, DR1.


Time-domain astronomy is a key component of LSST's four science pillars and is enabled by alerts on LSST detections of transient, variable, and/or moving objects.
Alerts are the only data product that will be immediately available (within 60 seconds of image readout) and publicly shareable, i.e. not subject to a proprietary period \citep{LSE-163},  \citep{RDO-013}.
The worldwide community is actively preparing to process the LSST alert stream and use it to generate groundbreaking scientific results. Additionally, for many science goals, time-sensitive follow-up observations after discovery are crucial to take full advantage of the Rubin data.


A key component of the Early Science Program is the capability to build {\it incremental} templates from on-sky imaging as it becomes available during commissioning and the early phases of the survey.
Such templates will be built periodically as images accumulate to allow for partial alert generation over an incomplete survey footprint.
Where possible, templates will be built from all available commissioning data before the start of year one and used to generate alerts during year one.
How extensive these templates are at the start of full survey operations will be influenced by the overall success of commissioning.
During year 1, templates will be built progressively from data obtained during year one (e.g., on a monthly timescale), and used to generate alerts during year one, either instead of, or in addition to using commissioning data to build templates.
